% Template for PLoS
% Version 3.5 March 2018
%
% % % % % % % % % % % % % % % % % % % % % %
%
% -- IMPORTANT NOTE
%
% This template contains comments intended
% to minimize problems and delays during our production
% process. Please follow the template instructions
% whenever possible.
%
% % % % % % % % % % % % % % % % % % % % % % %
%
% Once your paper is accepted for publication,
% PLEASE REMOVE ALL TRACKED CHANGES in this file
% and leave only the final text of your manuscript.
% PLOS recommends the use of latexdiff to track changes during review, as this will help to maintain a clean tex file.
% Visit https://www.ctan.org/pkg/latexdiff?lang=en for info or contact us at latex@plos.org.
%
%
% There are no restrictions on package use within the LaTeX files except that
% no packages listed in the template may be deleted.
%
% Please do not include colors or graphics in the text.
%
% The manuscript LaTeX source should be contained within a single file (do not use \input, \externaldocument, or similar commands).
%
% % % % % % % % % % % % % % % % % % % % % % %
%
% -- FIGURES AND TABLES
%
% Please include tables/figure captions directly after the paragraph where they are first cited in the text.
%
% DO NOT INCLUDE GRAPHICS IN YOUR MANUSCRIPT
% - Figures should be uploaded separately from your manuscript file.
% - Figures generated using LaTeX should be extracted and removed from the PDF before submission.
% - Figures containing multiple panels/subfigures must be combined into one image file before submission.
% For figure citations, please use "Fig" instead of "Figure".
% See http://journals.plos.org/plosone/s/figures for PLOS figure guidelines.
%
% Tables should be cell-based and may not contain:
% - spacing/line breaks within cells to alter layout or alignment
% - do not nest tabular environments (no tabular environments within tabular environments)
% - no graphics or colored text (cell background color/shading OK)
% See http://journals.plos.org/plosone/s/tables for table guidelines.
%
% For tables that exceed the width of the text column, use the adjustwidth environment as illustrated in the example table in text below.
%
% % % % % % % % % % % % % % % % % % % % % % % %
%
% -- EQUATIONS, MATH SYMBOLS, SUBSCRIPTS, AND SUPERSCRIPTS
%
% IMPORTANT
% Below are a few tips to help format your equations and other special characters according to our specifications. For more tips to help reduce the possibility of formatting errors during conversion, please see our LaTeX guidelines at http://journals.plos.org/plosone/s/latex
%
% For inline equations, please be sure to include all portions of an equation in the math environment.
%
% Do not include text that is not math in the math environment.
%
% Please add line breaks to long display equations when possible in order to fit size of the column.
%
% For inline equations, please do not include punctuation (commas, etc) within the math environment unless this is part of the equation.
%
% When adding superscript or subscripts outside of brackets/braces, please group using {}.
%
% Do not use \cal for caligraphic font.  Instead, use \mathcal{}
%
% % % % % % % % % % % % % % % % % % % % % % % %
%
% Please contact latex@plos.org with any questions.
%
% % % % % % % % % % % % % % % % % % % % % % % %

\documentclass[10pt,letterpaper]{article}
\usepackage[top=0.85in,left=2.75in,footskip=0.75in]{geometry}

% amsmath and amssymb packages, useful for mathematical formulas and symbols
\usepackage{amsmath,amssymb}

% Use adjustwidth environment to exceed column width (see example table in text)
\usepackage{changepage}

% Use Unicode characters when possible
\usepackage[utf8x]{inputenc}

% textcomp package and marvosym package for additional characters
\usepackage{textcomp,marvosym}

% cite package, to clean up citations in the main text. Do not remove.
% \usepackage{cite}

% Use nameref to cite supporting information files (see Supporting Information section for more info)
\usepackage{nameref,hyperref}

% line numbers
\usepackage[right]{lineno}

% ligatures disabled
\usepackage{microtype}
\DisableLigatures[f]{encoding = *, family = * }

% color can be used to apply background shading to table cells only
\usepackage[table]{xcolor}

% array package and thick rules for tables
\usepackage{array}

% create "+" rule type for thick vertical lines
\newcolumntype{+}{!{\vrule width 2pt}}

% create \thickcline for thick horizontal lines of variable length
\newlength\savedwidth
\newcommand\thickcline[1]{%
  \noalign{\global\savedwidth\arrayrulewidth\global\arrayrulewidth 2pt}%
  \cline{#1}%
  \noalign{\vskip\arrayrulewidth}%
  \noalign{\global\arrayrulewidth\savedwidth}%
}

% \thickhline command for thick horizontal lines that span the table
\newcommand\thickhline{\noalign{\global\savedwidth\arrayrulewidth\global\arrayrulewidth 2pt}%
\hline
\noalign{\global\arrayrulewidth\savedwidth}}


% Remove comment for double spacing
%\usepackage{setspace}
%\doublespacing

% Text layout
\raggedright
\setlength{\parindent}{0.5cm}
\textwidth 5.25in
\textheight 8.75in

% Bold the 'Figure #' in the caption and separate it from the title/caption with a period
% Captions will be left justified
\usepackage[aboveskip=1pt,labelfont=bf,labelsep=period,justification=raggedright,singlelinecheck=off]{caption}
\renewcommand{\figurename}{Fig}

% Use the PLoS provided BiBTeX style
% \bibliographystyle{plos2015}

% Remove brackets from numbering in List of References
\makeatletter
\renewcommand{\@biblabel}[1]{\quad#1.}
\makeatother



% Header and Footer with logo
\usepackage{lastpage,fancyhdr,graphicx}
\usepackage{epstopdf}
%\pagestyle{myheadings}
\pagestyle{fancy}
\fancyhf{}
%\setlength{\headheight}{27.023pt}
%\lhead{\includegraphics[width=2.0in]{PLOS-submission.eps}}
\rfoot{\thepage/\pageref{LastPage}}
\renewcommand{\headrulewidth}{0pt}
\renewcommand{\footrule}{\hrule height 2pt \vspace{2mm}}
\fancyheadoffset[L]{2.25in}
\fancyfootoffset[L]{2.25in}
\lfoot{\today}

%% Include all macros below

\newcommand{\lorem}{{\bf LOREM}}
\newcommand{\ipsum}{{\bf IPSUM}}


% Pandoc citation processing




\usepackage{forarray}
\usepackage{xstring}
\newcommand{\getIndex}[2]{
  \ForEach{,}{\IfEq{#1}{\thislevelitem}{\number\thislevelcount\ExitForEach}{}}{#2}
}

\setcounter{secnumdepth}{0}

\newcommand{\getAff}[1]{
  \getIndex{#1}{The Jackson Laboratory}
}

\providecommand{\tightlist}{%
  \setlength{\itemsep}{0pt}\setlength{\parskip}{0pt}}

\begin{document}
\vspace*{0.2in}

% Title must be 250 characters or less.
\begin{flushleft}
{\Large
\textbf\newline{Variation in epigenetic state correlates with gene
expression across nine inbred strains of
mice} % Please use "sentence case" for title and headings (capitalize only the first word in a title (or heading), the first word in a subtitle (or subheading), and any proper nouns).
}
\newline
% Insert author names, affiliations and corresponding author email (do not include titles, positions, or degrees).
\\
Catrina Spruce\textsuperscript{\getAff{The Jackson Laboratory}},
Anna L. Tyler\textsuperscript{\getAff{The Jackson Laboratory}},
Many more people\textsuperscript{\getAff{JAX-MG and JAX-GM}},
Gregory W. Carter\textsuperscript{\getAff{The Jackson
Laboratory}}\textsuperscript{*}\\
\bigskip
\textbf{\getAff{The Jackson Laboratory}}600 Main St.~Bar Harbor, ME,
04609\\
\bigskip
* Corresponding author: Gregory.Carter@jax.org\\
\end{flushleft}
% Please keep the abstract below 300 words
\section*{Abstract}
Abstract goes here.

% Please keep the Author Summary between 150 and 200 words
% Use first person. PLOS ONE authors please skip this step.
% Author Summary not valid for PLOS ONE submissions.
\section*{Author summary}
The author summary goes here if we submit to a journal that has one.

\linenumbers

% Use "Eq" instead of "Equation" for equation citations.
\hypertarget{abstract}{%
\section{Abstract}\label{abstract}}

It is well established that epigenetic features, such as histone
modifications and DNA methylation, are associated gene expression across
cell types. However, it is not well known how variation in genotype
affects epigenetic state, or to what extent such variation contributes
to variation in gene expression across genetically distinct individuals.
Here we investigated the relationship between heritable epigenetic
variation and gene expression in hepatocytes across nine inbred mouse
strains. Eight of the inbred strains were founders of the diversity
outbred (DO) mice, and the ninth was DBA/2J, which, along with C57Bl6/J,
is one of the founders of the BxD recombinant inbred panel of mice. We
surveyed four histone modifications, H3K4me1, H3K4me3, H3K27me3 and
H3K27ac, as well as DNA methylation. We used ChromHMM to identify 14
chromatin states representing distinct combinations of the four measured
histone modifications. We found that variation in chromatin state
mirrored genetic variation across the inbred strains. Furthermore,
epigenetic variation was correlated with gene expression across strains.
The correspondence between epigenetic state and gene expression was
replicated in an independent population of DO mice in which we imputed
local epigenetic state. In contrast, we found that DNA methylation did
not vary across inbred strains and was not correlated with variation in
expression in DO mice. This work suggests that chromatin state is highly
influenced by local genotype and may be a primary mode through which
expression quantitative trait loci (eQTLs) are mediated. We further
demonstrate that the mid-range resolution of chromatin states, between
that of SNPs and haplotypes paired with gene expression, is useful for
annotation of functional regions of the mouse genome. Finally, we
provide, to our knowledge, the first data resource to document variation
in chromatin state across genetically distinct individuals.

\hypertarget{introduction}{%
\section{Introduction}\label{introduction}}

Epigenetic modifications to DNA and its associated histone proteins
influence the accessibility of DNA to transcription machinery, and are
associated with up- and down-regulation of gene expression {[}26704082,
22641018, 22781841{]}. Across cell types, unique combinatorial patterns
of histone modifications mark chromatin states that establish cell
type-specific patterns of gene expression {[}20657582, 21441907{]}.
Similarly, the methylation of CpG sites around gene promoters and
enhancers influences transcription in a cell type-specific manner
{[}21701563, 20720541{]}.

These patterns of histone modifications and DNA methylation are
established during development. The result is a canonical epigenetic
landscape for coordination of major patterns of gene expression for each
cell type {[}sources about development{]}. As an organism ages and
responds to its environment, patterns of both histone modifications
{[}citation{]} and of DNA methylation change {[}citation{]}. Such
changes have been linked to scenescence {[}Horvath clock{]} and cancer
{[}citations{]}.

Epigenetic modifications coordinate the usage of a single genome to be
used for many different types of cells with diverse morphology and
physiology. This amazing feature of epigenetic modifications has been
intensely studied, and the variation in epigenetic landscapes across
cell types has been extensively documented {[}citations{]}. Less well
understood, however, is the role that genetic variation plays in
determining epigenetic landscapes.

Across genetically diverse populations of humans or mice, individual
cell types, such as hepatocytes, or cardiomyocytes, have globally
similar gene expression profiles that define their role within the
greater organism. However, it is also true that across individuals, gene
expression varies widely within the global constraints of cell type.
This variation can increase or decrease an organism's risk of developing
disease. Variation in gene expression has been extensively mapped to
variation in genetic loci, or expression quantitative trait loci (eQTL).
Large, coordinated efforts, such as the Genotype-Tissue Expression
(GTEx) Project {[}32913073, 32913075{]} have identified and catalogued
many such loci in humans, and countless independent studies have
identified eQTL in mice and other model organisms.

Although the link between genetic variation and gene expression has been
well studied, there is relatively little known about inter-individual
variation in epigenetic modifications, and how these variations are
related to variations in genotype and gene expression. The generation of
a more complete picture of inter-individual variation in epigenetic
modifications has the potential to increase our understanding of the
mechanisms of gene regulation, provide insights into the mechanisms
establishing cell type-specific epigenetic landscapes, and to improve
the functional annotation of the genome as it relates to the regulation
of gene expression. The vast majority of SNPs associated with human
disease traits are located in non-coding regions, suggesting that they
influence gene regulation, rather than protein function {[}citation{]}.
However, annotation of these regions is difficult without additional
genomic features, such as histone modifictions and DNA methylation.
Overlaying a map of variation in epigenetic features has the potential
to provide a picture of how genetic variation changes functional
elements, like enhancers and insulators, in the genome {[}citation{]}.

Advances in chromatin immunoprecipitation (ChIP) and sequencing
technologies now enable genome-wide surveys of histone modifications
with relatively few cells {[}20077036{]}, thus opening the door to the
possibility of cataloging epigenetic variation across cell types and
individuals. Here, we performed a survey of epigenetic variation in
hepatocytes across nine inbred mouse strains. We included the eight
founders of the Diversity Outbred/Collaborative Cross (DO/CC)
{[}citation{]} mice, as well as DBA/2J, which, along with C57Bl/6J, is
one of the founders of the widely used BxD recombinant inbred panel of
mice {[}citation{]}. We assayed four histone modifications (H3K4me1,
H3K4me3, H3K27me3, and H3K27ac), as well as DNA methylation. We used
ChromHMM {[}citation{]} to identify 14 chromatin states, classified by
unique combinations of the four histone marks, and investigated the
association between variation in these states and variation in gene
expression across the nine strains. We separately investigated the
relationship between DNA methylation and gene expression across strains.

We further investigated the relationship between epigenetic state and
gene expression by imputing the 14 chromatin states and DNA methylation
into a population of DO mice. We then mapped gene expression to the
imputed epigenetic states to assess the extent to which eQTLs are driven
by variation in epigenetic modification. We thus linked genetically
controlled variation in epigentic modifications to variation in gene
expression in mice, and we provide the first resource documenting
epigenetic variation across a wide panel of genetically diverse mice.

\hypertarget{materials-and-methods}{%
\section{Materials and Methods}\label{materials-and-methods}}

\hypertarget{inbred-mice}{%
\subsection{Inbred Mice}\label{inbred-mice}}

information about housing, animal use, etc.

\hypertarget{hepatocyte-acquisition}{%
\subsubsection{Hepatocyte acquisition}\label{hepatocyte-acquisition}}

Samples were taken from 12-week female mice of nine inbred mouse
strains: 129S1/SvImJ, A/J, C57BL/6J, CAST/EiJ, DBA/2J NOD/ShiLtJ,
NZO/HlLtJ, PWK/PhJ, and WSB/EiJ. Eight of these strains are the eight
strains that served as founders of the Collaborative Cross/Diversity
Outbred mice {[}REF{]}. The ninth strain, DBA/2J, will facilitate the
interpretation of existing and forthcoming genetic mapping data obtained
from the BxD recombinant inbred strain panel {[}REF{]}. Mice were aged
and processed in groups to maintain a steady sample preparation
workflow. Mice were housed, born, and aged in the same mouse room, with
uniformity in timing, diet, and all other possible conditions. Female
mice were used for all experiments due to potentially confounding
effects from variation in testosterone among males that can affect liver
gene expression, as well general experience that female expression is
less variable than male in multiple tissues. This will also facilitate
the analysis of maternal effects on offspring in later studies. Three
mice were used from each strain.

\hypertarget{liver-perfusion}{%
\subsubsection{Liver perfusion}\label{liver-perfusion}}

To purify hepatocytes from the liver cell population, the mouse livers
were perfused with collagenase to digest the liver into a single-cell
suspension, and then isolated using centrifugation. Mice were harvested
at 9:00 AM and sacrificed by cervical dislocation. Mice were placed over
a stack of paper towels in preparation to catch excess liquid, and the
appendages were pinned out to hold the body in place. to keep the fur
from contaminating the liver sample later, the fur was wiped down with
70\% ethanol. The mouse skin was then cut open and peeled back to the
appendages to allow clear access to the abdominal cavity. The fascia was
cut open and back to the ribs, being careful to not nick the liver.
Moving the intestines and stomach to the right side, the vena cava and
hepatic portal vein should be clearly visible below the liver.

For the perfusion, a 23G x \(\frac{3}{4}\)'\,' BD Vacutainer Safety-Lok
needle (REF 367297) was attached to 1.6mm ID BioRad Tygon tubing
(R-3603) connected to a Pharmacia peristaltic pump that allows a flow of
up to 8 ml/min. The liver will be processed with three solutions: 5mM
EGTA in Leffert's buffer, Leffert's buffer wash, and 87 CDU/mL Liberase
collagenase with 0.02\% CaCl2 in Leffert's buffer. The three solutions
were at 37\(^{\circ}\)C before perfusion.

The needle was placed into the vena cava for the perfusion superior to
the kidneys and inferior to the liver. With the peristaltic pump running
slowly, the vena cava was pierced at shallow 15\(^{\circ}\) angle and
the needle was inserted to a shallow depth (around 2-3mm of the needle
tip in the vena cava). Once the needle is inserted into the vena cava,
the volume on the peristaltic pump is increased to 5-7mL/min. The liver
will immediately blanch, and the hepatic portal vein is immediately
severed to allow flushing of the liver.

The 1x EGTA buffer was used to flush the blood out of the liver and
start the digestion of the desmosomes connecting the liver cells. To
help with the perfusion, pressure was applied to the hepatic portal vein
for 5 second intervals causing more solution to be forced through the
liver, which can be seen visually by the liver swelling. After 35ml of
the 1x EGTA solution is passed through the liver, the solution was
switched to the 1x Leffert's buffer. The pump was turned off during the
switch to prevent air from being sucked into the tubing while the tubing
is transferred to the new solution. To wash, 7-10ml of the Leffert's
buffer was passed through the liver to flush out the EGTA, which
otherwise chelates the calcium ions necessary for collagenase activity
in the next buffer. The pump was turned off again to switch to the
Liberase solution. To digest the liver, 25-50mL of Liberase solution
(\(\sim4.3\) wunsch units) was passed through the liver. Throughout the
perfusion process, periodic pressure was applied to the hepatic portal
vein to help pump the buffers more completely through the liver. As the
liver was digested with the Liberase, it will swell and look soggy and
limp. Over-digestion leads to increased contamination with
non-hepatocyte cell types, and further reduces cell viability.

After perfusion, which takes around 15-20min to complete, the liver was
carefully cut out of the abdominal cavity and placed in a petri dish
with 35 mL ice-cold Leffert's buffer with 0.02\% CaCl\(_{2}\). The
digested liver was passed through Nitex 80 \(\mu\)m nylon mesh (cat
\#03-80/37) into a 50mL conical, using additional ice-cold Leffert's
buffer with 0.02\% CaCl2 if necessary, and a rubber policeman. After the
liver cells from both animals were collected, they were put through two
wash and spin cycles to purify the hepatocytes and remove other types of
cells. To isolate the hepatocytes, the much larger size of the
hepatocyte cells was exploited in very slow 4 min, 50 x g spins that
leave smaller other cell types in suspension. After each spin, the
solution was decanted as waste, and the enriched cell pellet of
hepatocytes was resuspended in 30ml ice-cold Leffert's buffer with
0.02\% CaCl\(_{2}\). After the second spin, the solution should be
almost clear, indicating that other cell types have been removed. The
hepatocytes are resuspended in room temperature PBS, counted, and volume
adjusted to \(1 x 10^{6}\) cells/mL.

We aliquoted \(5 x 10^{6}\) cells for each RNA-Seq and bisulfite
sequencing, and the rest were cross-linked for ChIP assays. Two
\(5 x 10^6\) aliquots (5mLs) of liver cells were removed into two 15mL
conicals. These were spun down at 200 rpm for 5 min, and resuspended in
\(1200\mu L\) RTL+BME (for RNA-Seq) or frozen as a cell pellet in liquid
nitrogen (for bisulfite sequencing). Meanwhile, 37\% formaldehyde in
methanol (VENDOR) were added to the remaining cells to a final
concentration of 1\%. The cells were rotated at room temperature for 5
min to cross-link protein complexes to the DNA bound to them. After
cross-linking, 10x glycine (VENDOR) is added to a final concentration of
125 mM and rotated for 5 min to quench the formaldehyde and stop
cross-linking. The cells were spun down at 2000 rpm for 5 min, decanted,
and resuspended in PBS to \(5 x 10^6\) cells/mL. The cells were divided
into \(5x10^6\) aliquots in 2mL tubes. The tubes were spun down again at
5000 x g for 5 min, decanted, and the cell pellets frozen in liquid
nitrogen. All cell samples were stored at -80°C until used.

\hypertarget{hepatocyte-histone-binding-and-gene-expression-assays}{%
\subsubsection{Hepatocyte histone binding and gene expression
assays}\label{hepatocyte-histone-binding-and-gene-expression-assays}}

Hepatocyte samples from 30 treatment and control mice were used in the
following assays:

\begin{enumerate}
\def\labelenumi{\arabic{enumi}.}
\tightlist
\item
  RNA-seq to quantify mRNA and long non-coding RNA expression, with
  approximately 30 million reads per sample.
\item
  Reduced-representation bisulfate sequencing to identify methylation
  states of approximately two million CpG sites in the genome. The
  average read depth is 20-30x.
\item
  Chromatin immunoprecipitation and sequencing to assess binding of the
  following histone marks:
\end{enumerate}

\begin{enumerate}
\def\labelenumi{\alph{enumi}.}
\tightlist
\item
  H3K4me3 to map active promoters
\item
  H3K4me1 to identify active and poised enhancers
\item
  H3K27me3 to identify closed chromatin
\item
  H3K27ac, to identify actively used enhancers
\item
  A negative control (input chromatin) Samples are sequenced with
  \(\sim40\) million reads per sample.
\end{enumerate}

The samples for RNA-Seq in RTL+BME buffer were sent to The Jackson Lab
Gene Expression Service for RNA extraction and library synthesis.

\hypertarget{histone-chromatin-immunoprecipitation-assays}{%
\subsubsection{Histone chromatin immunoprecipitation
assays}\label{histone-chromatin-immunoprecipitation-assays}}

The H3K4me1 and H3K4me3 histone chromatin immunoprecipitation assays
were performed on cross-linked hepatocytes using similar protocols. For
all histone ChIP assays, the crosslinked chromatin was prepared the same
way. First, the aliquot of \(5x10^6\) hepatocyte cells was lysed to
release the nuclei by rotating the sample in hypotonic buffer for 20 min
at \(4^\circ\)C. The cells were pelleted by spinning for 10min, 10K x G,
at \(4^\circ\)C. The cells were resuspended in 130ul MNase buffer with
1mM PMSF (VENDOR) and 1x protease inhibitor cocktail (Roche VENDOR) to
prevent histone protein degradation, then digested with 15U of MNase.
The micrococcal nuclease digests the exposed DNA, but leaves the
nucleosome-bound DNA intact. After 10min of incubation at \(37^\circ\)C,
the chromatin was digested into primarily mononucleosomes. This was
confirmed by DNA-purification of the MNase-digested chromatin run out on
an agarose gel, which yielded mostly 150bp fragments, and few 300bp
fragments. The MNase digestion was stopped by adding EDTA to 10mM, and
incubating on ice for 5 min. The digested chromatin was purified by
spinning out insoluble parts at top speed for 10 min at \(4^\circ\)C.
The chromatin was transferred to a new tube and spun again to further
remove impurities and reduce background in the ChIP assays. The final
chromatin was transferred to a fresh tube, and used immediately in the
ChIP.

To prepare for the ChIP, \(20\mu L/1x10^6\) cells Dynabead Protein G
beads were aliquoted into an Eppendorf tube. A magnetic tube holder was
used to attract the beads to the wall of the tube, and then the solution
was carefully pipetted off, leaving only the beads behind. The beads
were washed twice with buffer to prepare them for binding to the
antibody. For this binding step and the chromatin binding step, the
buffer used was either RIPA buffer for the H3K4me3 and K3K27me3 ChIPs,
or ChIP buffer (VENDOR) for the H3K4me1 ChIP. The ChIP buffer was
gentler and less stringent than RIPA buffer, which was better for the
weaker binding of the H3K4me1 antibody that was used. The buffers were
supplemented with 50 mg/mL BSA (VENDOR) and 0.5 mg/mL Herring Sperm DNA,
both of which are blocking agents that reduce background and
non-specific binding. The ChIP assays also varied in the amount of input
chromatin and corresponding size of the reaction that was necessary to
yield sufficient DNA for sequencing. H3K4me3 ChIP needed only
\(1.5 x 10^6\) cells, and H3K4me1 and K3K27me3 ChIP used \(4 x 10^6\)
cells. To perform the ChIP, \(20\mu\)L of Dynabeads per \(1 x 10^6\)
cells is incubated with \(5\mu L\) of histone antibody for \(>20\)min in
\(50\mu L/1x10^6\) cells RIPA (or ChIP) buffer supplemented with 50
mg/mL BSA, 0.5 mg/mL Herring Sperm DNA, 1xPIC, and 1mM PMSF. The
antibodies used were (XXX). Once the antibody was bound to the
Dynabeads, the beads were washed twice with \(100\mu L/1x10^6\) cells
RIPA buffer with BSA and Herring Sperm DNA.

Next, the MNase-digested chromatin were added, which was at a
concentration of \(1x10^6\) cells/\(25\mu L\). The ChIP reaction was
incubated overnight with rotation at \(4^\circ\)C, to allow the histone
protein to bind to the antibody, which was bound to the magnetic beads.
In order to calculate enrichment for each ChIP sample, a known amount
(\(10\) or \(20\mu L\)) of MNase-digested input chromatin was saved.\\
The next morning, the ChIPs underwent a series of washes to remove
unbound chromatin. The H3K4me3 and H3K27me3 ChIPs were washed 3x with
\(100\mu L/1x10^6\) cells RIPA buffer, and the H3K4me1 ChIP was washed
with a low salt wash (0.1\% SDS, 1\% Triton X-100, 2mM EDTA, 20mM
Tris-HCl pH 8, 150 mM NaCl), a high salt wash (0.1\% SDS, 2\% Triton
X-100, 2mM EDTA, 20mM, Tris-HCl, pH 8, 500mM NaCl), and a LiCl wash
(0.25 MLiCl, 1\% IGEPAL-CA630, 1\% deoxycholic acid (sodium salt), 1 mM
EDTA, 10 mM Tris-HCl pH 8). After three washes, the ChIPs were washed
twice with TE buffer and transferred to a new tube during the last TE
wash to reduce background. At this point, the histone of interest and
the histone-bound DNA fragment had been purified from the
MNase-digested, cross-linked chromatin, and was bound by
histone-specific antibody to the magnetic Dynabeads. In the next step, a
high-salt elution buffer is used to degrade the antibody binding
interactions to the beads and the histone, and concurrently, proteinase
K is added to digest the protein away from the DNA-protein complexes.
The ChIP was incubated with the elution buffer and proteinase K at
\(68^\circ\)C for \(>6\) hours to liberate the DNA. At the same time,
the saved input chromatin was also digested in the same buffer.
Afterwards, the beads were removed using the magnet, and the DNA was
purified using the Qiagen PCR purification kit. Quantification was
performed using the Qubit quantification system, which is accurate to
\(0.02 ng/\mu L\) and only requires a small amount of sample to measure
concentration. The ChIP sample was enriched for only DNA that was bound
to the histone of interest. The goal for each ChIP was to yield 10 ng of
ChIP DNA for sequencing. Not all samples met this criterion, and the
H3K4me1 ChIPs often had a total yield of \(\sim 2 ng\) of DNA.

To test the efficiency of the ChIPs, quantitative PCR using QuantiFAST
was performed. Two sets of primers were used, one set in a known region
of histone binding (positive control), and one set in a region without
histone binding (negative control). The qPCR was performed both on the
ChIP DNA and the input DNA. Then the relative enrichment of positive vs
negative assays was compared between the ChIP and input DNA.

The ChIP DNA was submitted to The Jackson Lab GES service for library
preparation and sequencing. Libraries were made using the Kapa Hyper
Prep kit with adapters at \(0.6\mu M\). The libraries were amplified by
10 cycles of PCR. These libraries were not size selected, although most
fragments were \(\sim150\) bp due to MNase-digestion. The samples were
sequenced with 40 or more million reads per sample, which is almost 2x
more reads than the ENCODE project, which sequenced using 20 million
reads.

\hypertarget{diversity-outbred-mice}{%
\subsection{Diversity Outbred mice}\label{diversity-outbred-mice}}

We used previously published data from a population of diversity outbred
(DO) mice {[}Svenson et al.~2012{]} to compare to the data collected
from the inbred mice. The DO population included males and females from
DO generations four through 11. Mice were randomly assigned to either a
chow diet (6\% fat by weight, LabDiet 5K52, LabDiet, Scott Distributing,
Hudson, NH), or a high-fat, high-sucrose (HF/HS) diet (45\% fat, 40\%
carbohydrates, and 15\% protein) (Envigo Teklad TD.08811, Envigo,
Madison, WI). Mice were maintained on this diet for 26 weeks.

\hypertarget{genotyping}{%
\subsubsection{Genotyping}\label{genotyping}}

All DO mice were genotyped as described in Svenson et al.~(2012) using
the Mouse Universal Genotyping Array (MUGA) (7854 markers), and the
MegaMUGA (77,642 markers) (GeneSeek, Lincoln, NE). All animal procedures
were approved by the Animal Care and Use Committee at The Jackson
Laboratory (Animal Use Summary \# 06006).

Founder haplotypes were inferred from SNPs using a Hidden Markov Model
as described in Gatti\textsubscript{\textit{et~al.}}2014. The MUGA and
MegaMUGA arrays were merged to create a final set of evenly spaced
64,000 interpolated markers.

\hypertarget{tissue-collection-and-gene-expression}{%
\subsubsection{Tissue collection and gene
expression}\label{tissue-collection-and-gene-expression}}

At sacrifice, whole livers were collected and gene expression was
measured using RNA-Seq as described in (Chick, Munger et
al.\textasciitilde2016, and Tyler et al.\textasciitilde2017). Transcript
sequences were aligned to strain-specific genomes, and we used an
expectation maximization algorithm (EMASE) to estimate read counts
(\url{https://github.com/churchill-lab/emase}).

\hypertarget{data-processing}{%
\subsection{Data Processing}\label{data-processing}}

\hypertarget{sequencing}{%
\subsubsection{Sequencing}\label{sequencing}}

The raw sequencing data from both RNA-Seq and ChIP-Seq was put through
the quality control program FastQC. FastQC identifies problems or biases
in either the sequencer run or the starting library material. The FastQC
readout includes total number of reads, sequence quality, duplication
level, and overrepresented sequences. All of our samples had comparable
quality levels and no outstanding flags. However, the ChIP-Seq data was
flagged for having a high level of duplicate reads. This can be
explained by the use of MNase to shear the DNA into 150 bp fragments. If
the binding positions of nucleosomes are fixed, then the MNase enzyme
will cleave the DNA in the same place in multiple cells, resulting in
duplicate pieces of DNA. Despite evidence that the duplication rate has
a biological explanation, duplicates were removed before downstream
analysis, as is typical in sequencing workflows, to avoid potential
biases caused by starting libraries that have less diversity.

For the sequence analysis, reads from each sample were mapped to
strain-specific pseudogenomes that integrate known SNPs from each
strain. While the B6 samples were aligned directly to the reference
mouse genome, the other samples were from genetically different strains.
Strain-specific sequence variation in transcripts can affect alignment
quality and result in biased estimates of abundance. To counteract
potential strain biases, sequencing data from each strain were aligned
to a custom strain pseudogenome, allowing a more precise
characterization of gene expression and histone binding. The
pseudogenomes were created using the EMASE computational program
{[}REF{]} designed to construct customized genomes based on known SNP
and indel attributes. The resulting custom genomes are called
pseudogenomes, because they are based on inserting small known
variations into the reference genome, but do not attempt whole genome
sequencing for each strain and complete rebuild the entire genomic
sequence from the scaffold up. The strain-specific pseudogenomes were
then used in the Bowtie mapping algorithm to align and map reads from
the RNA-Seq and ChIP-Seq experiments.

\hypertarget{quantifying-gene-expression}{%
\subsection{Quantifying gene
expression}\label{quantifying-gene-expression}}

Once the sequencing data was mapped to the custom genomes, edgeR is used
to quantify transcripts. The edgeR program uses a Trimmed Mean of
M-values (TMM), which adjusts each sample for library size and RNA
composition using the assumption that most genes are not differentially
expressed. The output is sample read count for each of the ENSMUSG
transcript ID's. Next, transcripts with less than 1 CPM in two or more
replicates were filtered to remove lowly expressed genes. Also, the data
were trimmed to include only protein-coding transcripts.

\hypertarget{chip-seq-quantification}{%
\subsubsection{ChIP-Seq quantification:}\label{chip-seq-quantification}}

After the ChIP-Seq sequencing data were mapped to the custom
pseudogenomes, peaks were called in each sample using MACS 1.4.2
{[}18798982{]}, with a significance threshold of \(p \leq 10^{-5}\). In
order to compare peaks across strains, the MACS output peak coordinates
were converted to common B6 coordinates using g2g tools
(https://churchill-lab.github.io/g2gtools/).

\hypertarget{quantifying-dna-methylation}{%
\subsection{Quantifying DNA
methylation}\label{quantifying-dna-methylation}}

RRBS data were processed using a bismark-based pipeline modified from
Thompson et al.~2018 {[}30348905{]}. The pipeline uses Trim Galore!
0.6.3 (https://www.bioinformatics.babraham.ac.uk/projects/trim\_galore/)
for QC, followed by the trimRRBSdiversityAdaptCustomers.py script from
NuGen for trimming the diversity adapters. This script is available at:
https://github.com/nugentechnologies/NuMetRRBS

All of our samples had comparable quality levels and no outstanding
flags. Total number of reads was 45-90 million, with an average read
length of about 50 bp. Quality scores were mostly above 30 (including
error bars), with the average above 38. Duplication level was reduced to
\(<2\) for about 95\% of the sequences.

High quality reads were aligned to a custom strain pseudogenome, using
bowtie2 as implemented in Bismark 0.22 {[}21493656{]}. The pseudogenomes
were created by incorporating strain-specific SNPs and indels into the
reference genome using g2gtools
(https://github.com/churchill-lab/g2gtools), allowing a more precise
characterization of methylation patterns. Bismark methylation extractor
tool was then used for creating a bed file of estimated methylation
proportions for each animal, which was then translated to the reference
mouse genome (GRCm38) coordinates using g2gtools. Unlike other liftover
tools, g2gtools does not throw away alignments that land on indel
regions. B6 samples were aligned directly to the reference mouse genome.

\hypertarget{analysis}{%
\subsection{Analysis}\label{analysis}}

\hypertarget{filtering-transcripts}{%
\subsection{Filtering transcripts}\label{filtering-transcripts}}

For all gene expression data, we remove transcripts with extremely low
read counts, by filtering out those whose mean read count across all
individuals was less than five.

We used the R package sva {[}REF{]} to perform a variance stabilizing
transformation (vst) on the RNA-Seq read counts from both inbred and
outbred mice. In the inbred mice we used a blind transformation, while
in the outbred mice, we included DO wave and sex in the model. For eQTL
mapping, we performed rank Z normalization on the RNA-Seq read counts
across transcripts from the outbred mice.

\hypertarget{analysis-of-histone-modifications}{%
\subsection{Analysis of histone
modifications}\label{analysis-of-histone-modifications}}

\hypertarget{identification-of-chromatin-states}{%
\subsubsection{Identification of chromatin
states}\label{identification-of-chromatin-states}}

We used ChromHMM {[}29120462{]} to identify chromatin states, which are
unique combinations of the four chromatin modifications, for example,
high levels of both H3K4me3 and H3K4me1, and low levels of the other two
modifications. We conducted all subsequent analyses at the level of the
chromatin state.

To ensure we were analyzing the most biologically meaningful chromatin
states, we calculated chromatin states for all numbers of states between
four and 16, which is the maximum number of states possible with four
binary chromatin modifications (\(2^n\)). We aligned states across the
models by assigning each to one of the sixteen possible binary states
using an emissions probability of 0.3 as the threshold for
presence/absence of the histone mark. We then investigated the stability
of three features across all states: the emissions probabilities (Supp
Fig1), the abundance of each state across transcribed genes (Supp Fig2),
and the effect of each state on transcription (Supp Fig3). Methods for
each of these analyses are described separately below. All measures were
remarkably consistent across all models, but the 14-state model was
characterized by a wide range of relatively abundant states with
relatively strong effects on expression. We used this model for all
subsequent analyses.

\hypertarget{genome-distribution-of-chromatin-states}{%
\subsubsection{Genome distribution of chromatin
states}\label{genome-distribution-of-chromatin-states}}

We investigated genomic distributions of chromatin states in two ways.
First, we used the ChromHMM function OverlapEnrichment to calculate
enrichment of each state around known functional elements in the mouse
genome. We analyzed the following features:

\begin{itemize}
\tightlist
\item
  \textbf{Transcription start sites (TSS)} - Annotations of TSS in the
  mouse genome were provided by RefSeq {[}26553804{]} and included with
  the release of ChromHMM, which we downloaded on December 9, 2019
  {[}29120462{]}.
\item
  \textbf{Transcription end sites (TES)} - Annotations of TES in the
  mouse genome were provided by RefSeq and included with the release of
  ChromHMM.
\item
  \textbf{Transcription factor binding sites (TFBS)} - We downloaded
  TFBS coordinates from OregAnno {[}26578589{]} using the UCSC genome
  browser {[}12045153{]} on May 4, 2021.
\item
  \textbf{Promoters} - We downloaded promoter coordinates provided by
  the eukaryotic promoter database {[}27899657,25378343{]}, through the
  UCSC genome browser on April 26, 2021.
\item
  \textbf{Enhancers} - We downloaded annotated enhancers provided by
  ChromHMM through the UCSC genome browser on April 26, 2021.
\item
  \textbf{Candidates of cis regulatory elements in the mouse genome
  (cCREs)} - We downloaded cCRE annotations provided by ENCODE
  {[}22955616{]} through the UCSC genome browser on April 26, 2021.
\item
  \textbf{CpG Islands} - Annotations of CpG islands in the mouse genome
  were included with the release of ChromHMM.
\end{itemize}

In addition to these enrichments around individual elements, we also
calculated chromatin state abundance relative to the main anatomical
features of a gene. For each transcribed gene, we normalized the base
pair positions to the length of the gene such that the transcription
start site (TSS) was fixed at 0, and the transcription end site (TES)
was fixed at 1. We also included 1000 bp upstream of the TSS and 1000 bp
downstream of the TES, which were converted to values below 0 and above
1 respectively.

To map chromatin states to the normalized positions, we binned the
normalized positions into 41 bins defined by the sequence from -2 to 2
incremented by 0.1. If a bin encompassed multiple positions in the gene,
we assigned the mean value of the feature of interest to the bin. To
avoid potential contamination from regulatory regions of nearby genes,
we only included genes that were at least 2kb from their nearest
neighbor, for a final set of 14048 genes.

\hypertarget{chromatin-state-and-gene-expression}{%
\subsubsection{Chromatin state and gene
expression}\label{chromatin-state-and-gene-expression}}

We calculated the effect of each chromatin state on gene expression. We
did this both across genes and across strains. The across-gene analysis
identified states that are associated with high expression and low
expression within the hepatocytes independent of strain. The
across-strain analysis investigated whether variation in chromatin state
across strains contributed to variation in gene expression across
strains.

For each transcribed gene, we calculated the proportion of the gene body
that was assigned to each chromatin state. We then fit a linear model
separately for each state to calculate the effect of state proportion
with gene expression:

\begin{equation*}{\label{eqn:chromatin_effect}}
y_{e} = \beta x_{s} + \epsilon
\end{equation*}

where \(y_{e}\) is the rank Z normalized gene expression of the full
transcriptome in a single inbred strain, and \(x_{s}\) is the rank Z
normalized proportion of each gene that was assigned to state \(s\). We
fit this model for each strain and each state to yield one \(\beta\)
coefficient with 95\% confidence interval. The effects were not
different across strains, so we averaged the effects and confidence
intervals across strains to yield one summary effect for each state.

To calculate the effect of each chromatin state across strains, we first
standardized transcript abundance across strains for each transcript. We
also standardized the proportion of each chromatin state for each gene
across strains. We then fit the same linear model, where \(y_{e}\) was a
rank Z normalized vector concatenating all standardized expression
levels across all strains, and \(x_{s}\) was a rank Z normalized vector
concatenating all standardized state proportions across all strains. We
fit the model for each state independently yielding a \(\beta\)
coefficient and 95\% confidence interval for each state.

In addition to calculating the effect of state proportion across the
full gene body, we also performed the same calculations in a
position-based manner. This second analysis yielded an effect of each
state at multiple points along the gene body and a more nuanced view of
the effect of each state.

\color{lightgray}

\hypertarget{selecting-the-most-biologically-meaningful-model}{%
\subsubsection{Selecting the most biologically meaningful
model}\label{selecting-the-most-biologically-meaningful-model}}

We performed the above analyses on all states from the four-state model
to the 16-state model to find the most meaningful clustering of histone
modifications. Across all models, the states were remarkably stable. As
we increased the number of states detected by the model, new states
appeared, but previously detected states were not disrupted. This
stability was apparent in all state measures: emissions probability
(Supp Fig 1) patterns, overall abundance (Supp. Fig. 2), and effect on
expresssion (Supp Fig 3). This analysis revealed interesting patterns in
the detected states. For example, one highly abundant state (present in
65\% of transcribed genes) detected first in the four-state model was
split into two distinct states in the 10-state model. These resulting
states were also highly abundant (appearing in 40\% and 41\% of
transcribed genes), and had distinct emissions probabilities (Supp. Fig.
1). These two states remained stable with increasing numbers of clusters
through to the 16-state model. States arising after the 10-state model
were of lower abundance, appearing in 2\% or less of transcribed genes.

All of the higher abundance states were established in the 10-state
model. However, as we moved toward higher numbers of clusters, the
resolution on the lower-abundance states improved in terms of the
emissions probabilities profiles, and strength of the correlation with
gene expression. For example, the 14-state model better resolved a state
that had appeared in the 10-state model but was not strongly correlated
with gene expression. In the 14-state model, the emission patterns were
closer to binary, and the strength of the correlation with expression
was increased. Beyond 14 clusters, the new states identified were
extremely rare (1\% of transcripts or less), and were not strongly
correlated with gene expression. We thus selected the 14-state model and
the model with the most biologically meaningful clusters. \color{black}

\hypertarget{analysis-of-dna-methylation}{%
\subsection{Analysis of DNA
methylation}\label{analysis-of-dna-methylation}}

\hypertarget{creation-of-dna-methylome}{%
\subsubsection{Creation of DNA
methylome}\label{creation-of-dna-methylome}}

We combined the DNA methylation data into a single methylome cataloging
the methylated sites across all strains. For each site, we averaged the
percent methylation across the three replicates in each strain. The
final methylome contained 5,311,670 unique sites across the genome.
Because methylated CpG sites can be fully methylated, unmethylated, or
hemi-methylated, we rounded the average percent methylation at each site
to the nearest 0, 50, or 100.

\hypertarget{distribution-of-cpg-sites}{%
\subsubsection{Distribution of CpG
sites}\label{distribution-of-cpg-sites}}

We used the enrichment function in ChromHMM described above to identify
enrichment of CpG sites around functional elements in the mouse genome.
We further performed a gene-based analysis of abundance similar to that
in the chromatin states. As a function of relative position on the gene
body, we calculated the density of CpG sites as the average distance to
the next downstream CpG site, as well as the percent methylation at each
site.

\hypertarget{effects-of-dna-methylation-on-gene-expression}{%
\subsubsection{Effects of DNA methylation on gene
expression}\label{effects-of-dna-methylation-on-gene-expression}}

As with chromatin state, we assessed the effect of DNA methylation on
gene expression both across genes and across strains. We used the same
linear model described above, except that \(y_{s}\) became the rank Z
normalized percent methylation either across genes or across strains.
Because the effect of DNA methylation on gene expression is well-known
to be dependent on position, we only calculated a position-dependent
effect on expression.

\hypertarget{imputation-of-genomic-features-in-diversity-outbred-mice}{%
\subsection{Imputation of genomic features in Diversity Outbred
mice}\label{imputation-of-genomic-features-in-diversity-outbred-mice}}

To assess the extent to which chromatin state and DNA methylation are
responsible for local expression QTLs, we imputed local chromatin state
and DNA methylation into a population of diversity outbred (DO) mice
described above and in Svenson et al.~2012. We compared the effect of
the imputed epigenetic features to imputed SNPs.

All imputations followed the same basic procedure: For each transcript,
we identified the haplotype probabilities in the DO mice at the genetic
marker nearest the gene transcription start site. This matrix held DO
individuals in rows and DO founder haplotypes in columns (Supp. Fig.
\ref{supp_fig:imputation}).

For each transcript, we also generated a three-dimensional array
representing the genomic features derived from the DO founders. This
array held DO founders in rows, feature state in columns, and genomic
position in the third dimension. The feature state for chromatin
consisted of states one through 14, for SNPs feature state consisted of
the genotypes A,C,G, and T.

We then multiplied the haplotype probabilities by each genomic feature
array to obtain the imputed genomic feature for each DO mouse. This
final array held DO individuals in rows, the genomic feature in the
second dimension, and genomic position in the third dimension. This
array is analagous to the genoprobs object in R/qtl2 {[}30591514{]}. The
genomic position dimension included all positions from 1 kb upstream of
the TSS to 1 kb downstream of the TES. SNP data for the DO founders in
mm10 coordinates were downloaded from the Sanger SNP database
{[}1921910, 21921916{]}, on July 6, 2021.

To calculate the effect of each imputed genomic feature on gene
expression in the DO population, we fit a linear model. From this linear
model, we calculated the variance explained (\(R^2\)) by each genomic
feature, thereby relating gene expression in the DO to each position of
the imputed feature in and around the gene body.

\hypertarget{results}{%
\section{Results}\label{results}}

Gene expression varies widely and reproducibly across inbred strains of
mice. This is seen as a clustering of individuals from the same strain
in a principal component plot of the hepatocyte transcriptome across
strains (Figure \ref{fig:pc_plots}A). Patterns of DNA methylation
(Figure \ref{fig:pc_plots}B) and individual histone modifications
(Figure \ref{fig:pc_plots}C-F) cluster in a similar pattern. This
suggests that these epigenetic features may relate to gene expression in
a manner that is consistent with genetic background.

\hypertarget{chromatin-state-overview}{%
\subsection{Chromatin state overview}\label{chromatin-state-overview}}

To investigate this association, we used ChromHMM to identify 14
chromatin states composed of unique combinations of four histone
modifications in the hepatocytes of nine inbred strains of mice. Panel A
in Figure \ref{fig:state_overview} shows the representation of each
histone modification across the states.

The states were distributed non-randomly around known functional
elements in the mouse genome (Figure \ref{fig:state_overview}B). The
majority of the states were enriched around the TSS, and other
TSS-related functional elements, such as promoters and CpG islands. Two
states (states 2 and 1) were primarily found in intergenic regions.
Three states (states 9, 13, and 11) were enriched around known
enhancers, and one (state 6) was enriched predominantly near the TES.
The majority of these states were also associated with variation in gene
expression. The colored bars in Figure \ref{fig:state_overview}C) show
the effect of each state on gene expression across the inbred strains.
For reference, the paired tan bars show the effect of each chromatin
state on gene expression in hepatocytes. These effects tend to be of the
same sign and greater magnitude than the across-strain effects.

The states in Figure \ref{fig:state_overview} are shown in order of
their effect on expression, which helps illustrate several patterns in
the data. The state with the largest negative effect on gene expression,
state 1, is the absence of all measured modifications. The next few
states all contain the repressive mark H3K27me3, and are all associated
with reduced gene expression. The states with the largest positive
effects on expression all have some combination of the activating marks,
H3K4me3, H3K4me1, and H3K27ac. The repressive mark is less commonly seen
in these activating states.

By merging the information from Figure \ref{fig:state_overview}A-C), we
were able to suggest annotations for many of the 14 chromatin states
(Figure \ref{fig:state_overview}D). States with the strongest effects on
expression had the clearest annotations, while states with weaker
effects remained unannotated.

\hypertarget{spatial-distribution-of-epigenetic-modifications-around-gene-bodies}{%
\subsection{Spatial distribution of epigenetic modifications around gene
bodies}\label{spatial-distribution-of-epigenetic-modifications-around-gene-bodies}}

In addition to looking for enrichment of chromatin states near annotated
functional elements, we characterized the fine-grained spatial
distribution of each state around gene bodies (Figure
\ref{fig:state_abundance}A-B). We similarly characterized the
distribution of CpG sites and their percent methylation at this
gene-level scale (Figure \ref{fig:state_abundance}C-D).

The spatial patterns of the individual chromatin states are shown in
(Figure \ref{fig:state_abundance}A), and an overlay of all states
together (Figure \ref{fig:state_abundance}B) emphasizes the difference
in abundance between the most abundant states (states 14, 12, and 1),
and the remaining states, which were relatively rare.

Each chromatin state had a characteristic distribution pattern relative
to gene bodies. For example, state 1, which was characterized by the
absence of all measured histone modifications, was strongly depleted
near the TSS, indicating that this region is commonly subject to histone
modification. However, its abundance increased steadily through the gene
to a peak at the TES. In contrast, states 12 and 14 were both
concentrated at the TSS. State 12 was very narrowly concentrated right
at the TSS, whereas state 14 was more broadly abundant both upstream and
downstream of the TSS. Both were associated overall with increased
expression in the inbred mice (indicated by red shading), suggesting
promoter or enhancer functions. The third state in this group of
high-expressing states, state 13, was depleted nere the TSS, but
enriched within the gene body, suggesting that this state may mark
active intragenic enhancers.

States with weaker effects on expression (indicated by grayer shades)
were of lower abundance. However, they still had distinct distribution
patterns around the gene body suggesting the possibility of distinct
functional roles in the regulation of gene expression.

There were similarly dramatic spatial patterns in DNA methylation
(Figure \ref{fig:state_abundance}C-D). Across all genes, the TSS had
densely packed CpG sites relative to the gene body (Figure
\ref{fig:state_abundance}C). As expected, the median CpG site near the
TSS was consistently hypomethylated relative to the median CpG site in
intergenic regions (Figure \ref{fig:state_abundance}D). CpG sites within
the gene body were slightly hypermethylated compared to intergenic CpGs.

\hypertarget{spatially-resolved-effects-on-gene-expression}{%
\subsection{Spatially resolved effects on gene
expression}\label{spatially-resolved-effects-on-gene-expression}}

The distinct spatial distributions of the chromatin states and
methylated CpG sites around the gene body raised the question as to
whether the effects of these states on gene expression could also be
spatially resolved. To investigate this possibility we tested the
association between both chromatin state and DNA methylation and gene
expression with spatially resolved models (Methods). We tested the
effect of each chromatin state on expression across genes within
hepatocytes (Figure \ref{fig:state_effects}A) and the effect of each
chromatin state on the variation in gene expression across strains
(Figure \ref{fig:state_effects}B).

All chromatin states demonstrated spatially dependent effects on gene
expression within hepatocytes. For many of the states, the effects on
expression were concentrated at or near the TSS, while in the other
states effects were seen across the whole gene. The direction of the
effects matched the overall effects of each state seen previously
(Figure \ref{fig:state_overview}). Remarkably, the spatial effects were
recapitulated for almost every state when we measured across strains.
That is, variation in chromatin state across strains contributed to
variation in gene expression in the same manner that cell-type
expression was being established. One notable exception was state 9,
whose presence upregulated genes within hepatocytes, but did not
contribute to expression variation across strains.

We also examined the effect of percent DNA methylation across genes
within hepatocytes, and across strains (Figure
\ref{fig:DNA_methylation_effect}). As expected, hypomethylation at the
TSS was associated with lower expression in hepatocytes. However,
percent DNA methylation did not contribute at all to expression
variation across strains, implying that although percent DNA methylation
is used in gene regulation within a cell type, it is not heritable and
does not contribute to variation in gene expression across genetically
diverse individuals.

\hypertarget{imputed-chromatin-state-explained-expression-variation-in-diversity-outbred-mice}{%
\subsection{Imputed chromatin state explained expression variation in
diversity outbred
mice}\label{imputed-chromatin-state-explained-expression-variation-in-diversity-outbred-mice}}

Thus far, we have used inbred strains of mice to identify correlations
between local chromatin state and gene expression. However, we cannot
establish causality in this population. For that we need a mapping
population in which we can associate genetic or epigenetic variation at
a single locus with changes in gene expression. A mapping population
also allows us to establish the extent to which variation in epigenetic
factors contributes to observed expression quantitative trait loci
(eQTL).

To compare the contribution of genetic and epigenetic features to eQTLs
in a gentically diverse population, we imputed chromatin state, DNA
methylation, and SNPs into a population of DO mice described previously
{[}Svenson, Tyler{]} (Methods). Chromatin state is largely determined by
local genotype, especially early in life {[}REF{]}, and can thus be
reliably imputed from local genotype. Further, we have shown here that
local chromatin state correlates with variation in gene expression
across inbred strains. DNA methylation, on the other hand, is known not
to be highly heritable {[}REF{]}, and thus cannot be reliably imputed
from local genotype. We have also shown here that DNA methylation is not
correlated with variation in gene expression across inbred strains. The
imputation of DNA methylation thus serves as an estimate of a lower
bound the ability of a feature imputed from local haplotype to explain
gene expression in a new population.

For each transcript in the DO population, we imputed the local chromatin
state across the gene body based on the gene's local founder haplotype
and the chromatin state at the corresponding position in the inbred
mice. We did the same for DNA methylation and SNPs.

After imputing each genomic feature into the DO population, we mapped
gene expression to the imputed features and calculated the variance
explained. Examples of each genomic feature and the mapping results for
the gene \textit{Pkd2} are shown in Figure \ref{fig:example_gene}. There
are two particularly interesting regions in this gene. One is at the TSS
and the immediately surrounding area, and the other is just downstream
of the TSS.

These two regions are colored red, indicating that they are marked by
chromatin states with a positive effect on gene expression. The order of
the rows in this panel helps illustrate that the strains with the most
red in chromatin state space contributed the highest-expressing alleles
to the DO (Figure \ref{fig:example_gene}E). The two haplotypes with the
strongest negative effect on gene expression in the DO have mostly blue
chromatin states in these two regions. These two strains also had the
lowest expression among the inbred mice (Figure
\ref{fig:example_gene}F). The concordance between chromatin state and
gene expression in the DO is seen as the blue pluses in Figure
\ref{fig:example_gene}A that are aligned with the two red regions, which
we suggest are putative enhancer regions.

The spatial patterns in the SNPs only partially mirror those in
chromatin state (Figure Figure \ref{fig:example_gene}C). SNPs underlying
the putative enhancer regions could potentially influence gene
expression by altering chromatin state. But SNPs downstream of this
region underly invariant chromatin.

Percent DNA methylation does not vary across the strains in either of
these putative enhancer regions, and does not contribute to variation in
expression across genetically distinct individuals (Figure
\ref{fig:example_gene}D).

The overall distributions of variance explained by each feature across
all transcripts is shown in Figure \ref{fig:effect_distrubutions}. These
distributions show the haplotype effect for the marker nearest each
transcript compared with the maximum effect across the gene body for
each of the other imputed features. Overall, local haplotype explained
the largest amount of variance of gene expression in the DO
(\(R^2 = 0.17\)). The variance explained by local chromatin state was
very highly correlated with that of haplotype (Pearson \(r = 0.96\)) and
explained almost as much variance in gene expression in the DO as local
haplotype (\(R^2 = 0.15\)).

The mean variance explained by SNPs was lower (\(R^2 = 0.13\)) than that
explained by haplotype and was not as highly correlated with local
haplotype as chromatin state was (Pearson \(r = 0.93\)). DNA
methylation, the lower bound for variance explained by a feature imputed
from local haplotype, explained the lowest amount of expression variance
in the DO population (\(R^2 = 0.09\)), and had a much lower correlation
to haplotype than either chromatin state or SNPs (Pearson \(r = 0.74\)).

\hypertarget{discussion}{%
\section{Discussion}\label{discussion}}

In this sudy we showed that variation in histone modifications in inbred
mice mirrors genetic variation, and we further showed that this
variation was highly related to variation in gene expression across
strains. These observations suggest that cell type-specific patterns of
histone modifications are determined by local genotype, and may be a
major mechanism through which expression QTL (eQTL) are generated. This
hypothesis was supported by the high concordance between chromatin
state, which was imputed from local genotype, and gene expression in an
independent outbred population of mice.

The high resolution of the chromatin states combined with spatial
patterns of abundance and effect on gene expression offers opportunities
for the annotation of functional elements in and around genes. For
example, the chromatin state patterns in the gene \textit{Pkd2}, suggest
two enhancers -- one at the TSS, and the other just downstream of the
TSS inside the gene body. The positive effects of these putative
enhancer regions in the inbred mice were replicated in outbred mice
suggesting that these effects are robust and contribute to variation in
gene expression seen in diverse populations.

The putative enhancers are not apparent in the SNP patterns or in the
patterns or DNA methylation, which suggests that chromatin modification
is the primary mechanism through which gene expression is regulated by
these regions. Further, the richness of the information in this
chromatin state layer provides data with which to further annotate the
effects of SNPs underlying these regions. There are SNPs throughout the
gene, as seen in Figure \ref{fig:example_gene}, and many of them are
associated with variation in gene expression. However, while the SNPs
within the putative enhancer regions may change expression by altering
histone modifications placed in those regions, SNPs futher downstream
may work through another mechanism, such as through directly dirsupting
transcription, or by altering the transcript such that it is processed
differently post transcriptionally. The intermediate resolution of the
chromatin state between that of SNPs and haplotype thus provides a
highly informative layer of information between genotype and gene
expression.

In contrast to chromatin state, percent DNA methylation was not
associated with variation in gene expression across inbred strains or in
the outbred population. This was largely due to a lack of variation in
methylation across strains. An example of this observation is shown in
panel D of Figure \ref{fig:example_gene}. Despite strain variation in
both genotype and chromatin state at the TSS of \textit{Pkd2}, DNA
methylation is invariant -- the CpG island at the TSS is unmethylated in
all strains. Thus, although chromatin state appears to be highly
influenced by local genotype, percent DNA methylation is not.

Similar observations have been made in human studies {[}33931130{]}.
Multiple twin studies have estimated the average heritability of
individual CpG sites to be roughly 0.19 {[}27051996, 24183450,
22532803{]}, with only about 10\% of CpG sites having a heritability
greater than 0.5 {[}24183450, 22532803, 24887635{]}. Trimodal CpG sites,
i.e.~those with methylation percent varying among 0, 50, and 100\%, have
been shown in human brain tissue to be more heritable than unimodal, or
bimodal sites (\(h^2 = 0.8 \pm 0.18\)), and roughly half were associated
with local eQTL {[}20485568{]}. Here, we did not see an association
between trimodal CpG sites and gene expression across strains
(Supplemental Figure XXX).

The diversity in the effects observed in the 14 chromatin states
highlights the importance of analyzing combinatorial states as opposed
to individual histone modifications. To illustrate this point, consider
the three states with the largest positive effects on transcription.
Each of these three states had a distinct combination of the three
histone marks associated with transcriptional activation: H3K4me1,
H3K4me3, and H3K27ac. State 12 was characterized by high levels of
H3K4me3 and H3K27ac, and low levels of H3K4me1. State 13 was
characterized by high levels of H3K4me1 and H3K27ac, and low levels of
H3K4me3. And state 14 was characterized by high levels of all three
activating marks (Figure XXX). Although all three states were associated
with increased gene expression, each had a completely distinct spatial
distribution. State 12 was distributed in a very narrow band centered on
the TSS, while state 14 was distributed across a much broader region
centered upstream of the TSS. State 13 had a completely different
distribution -- it was depleted at the TSS, and most abundant within the
gene body and near the TES. This variation in spatial distribution was
mirrored in the spatial effects on transcription. State 12, which we
annotated as an active promoter, was positively associated with
transcription when it was present at the TSS. In contrast, states 13 and
14, which we annotated as enhancers, were associated with increased
transcription when present anywhere in the gene body (Figure XXX). We
would not be able to detect such patterns if analyzing the histone
modifications in isolation. These results highlight the complexity of
the histone code and the importance at analyzing combinatorial states.

While we were able to annotate several states, particularly those with
the strongest effects on gene expression, other states were more
difficult to annotate. This raises the intruiguing possibility of
identifying new modes of expression regulation through histone
modification. One of these unannotated states, state 6, had a weak, but
consistent negative effect on gene transcription centered within the
gene body, downstream of the TSS. This state was characterized by high
levels of H3K4me3 and low levels of the other three modifications.

The modification H3K4me3 is most frequently associated with increased
transcriptional activity {[}citation{]}, so the association with state 6
with reduced transcription is a deviation from the dominant paradigm.
The physical distribution of this state is also interesting. It was
depleted at the TSS, and enriched just upstream and just downstream of
the TSS (Figuree XXX). It was also enriched just downstream of the TES,
although it did not appear to influence transcription at this location
(Figure XXX). The group of genes marked by state 6 were enriched for
functions such as stress response, DNA damage repair, and ncRNA
processing suggesting that this state may be used to regulate subsets of
genes involved in responses to environmental stimuli.

There were other states that we were able to annotate, but were not
necessarily expecting to see in this study. We detected two bivalent
states, which are states that combine an activating histone modification
and a repressing histone modificaction and are usually associated with
undifferentiated cells {[}citation{]}. Here we identified two bivalent
states in adult mouse hepatocytes, and annotated them as a poised
enhancer (state 3) and a bivalent promoter (state 4). Both states were
associated with downregulation across inbred strains when present near
the TSS; however this effect was not replicated in the outbred mice. The
lack of replication was perhaps because the effect was too weak to
detect given the number of animals in the population.

Both bivalent promoters and poised enhancers are dynamic states that
change over the course of differentiation and in response to external
stimuli {[}citation{]}. Bivalent promoters have been studied primarily
in the context of development. They are abundant in undifferentiated
cells, and are typically resolved either to active promoters or to
silenced promoters as the cells differentiate into their final state
{[}23788621, 22513113{]}. These promoters have also been shown to be
important in the response to changes in the environment. Their abundance
increases in breast cancer cells in response to hypoxia {[}27800026{]}.
Poised enhancers are also observed during differentiation and in
differentiated cells {[}32432110{]}. In concordance with these previous
observations, the genes marked by states 3 and 4 were enriched for
vascular development and morphogenesis. That we identified these states
in differentiated hepatocytes may indicate that a subset of
developmental genes retain the ability to be activated under certain
circumstances, such as during liver regeneration in response to damage.
It is also possible these states were induced in the inbred strains in
respose to stress, rather than genetically coded. This could explain why
the negative effect on gene expression was not replicated in the outbred
mice. However, given that we detected this state in all nine inbred
strains in relatively equal proportions, this latter hypothesis seems
less likely.

Broadly, local variation in chromatin state was highly correlated with
variation in gene expression across individuals, an observation that was
replicated in an independent population of genetically diverse, outbred
mice. The percent variance explained by chromatin state closely matched
that of haplotype, and exceeded that of individual SNPs. These results
suggest two things. First, a large portion of the effect of local
haplotype on gene expression in mice is likely mediated through
variation in chromatin state. Second, the intermediate resolution of
chromatin state between that of individual SNPs and broad haplotypes
carries important imfornation that cannot be resolved at the other
levels. Individual SNPs, although, sometimes causally linked to trait
variation, are highly redundant and cannot be readily used to annotate
functional elements in the genome. Haplotypes aggregate genomic
information over broad regions and are a powerful tool to link genomic
variation to trait variation. However, they are usually too broad to be
used to annotate regions less than a few megabases in length. By
combining the mapping power of haplotypes, the high resolution of SNPs,
and the intermediate resolution of chromatin states, we can begin to
build mechanistic hypotheses that link genetic variation to variation in
physiology. Understanding the role that genetic variation plays in
modifying the chromatin state landscape will be critical in making these
links. Through this survey we are providing one of the first rigorous
resources that explores the connection between genetic variation and
epigenetic variation.

work this paragraph in\ldots{} That states 1 and 2 were associated with
reduced gene expression both within hepatocytes and across strains
suggests that there may be differential epigenetic silencing of genes in
hepatocytes across strains. Further, the majority of chromatin states
were associated with variation in expression across strains, suggesting
that epigenetic regulation of gene expression through histone
modification may contribute substantially to variation in gene
expression across genetically distinct individuals. That most states
have the same effects across genes within a cell type and across strains
suggests that the mechanisms that are used to regulate cell type
specificity also contribure to variation in genetically distinct
individuals.

\hypertarget{acknowledgements}{%
\section{Acknowledgements}\label{acknowledgements}}

This work was funded by XXX.

\hypertarget{data-and-software-availability}{%
\section{Data and Software
Availability}\label{data-and-software-availability}}

All data used in this study and the code used to analyze it are avalable
as part of a reproducible workflow located at\ldots{} (Figshare?,
Synapse?).

\hypertarget{figure-legends}{%
\section{Figure Legends}\label{figure-legends}}

\begin{figure}[ht]
\centering
\caption{The first two principle components of each genomic feature across
nine inbred strains of mouse. In all panels each point represents
an individual mouse, and strain is indicated by color as shown in
the legend at the bottom of the figure. Each panel is labeled with
the data used to generate the PC plot. (A) Hepatocyte transcriptome - 
all transcripts sequenced in isolated hepatocites. (B) DNA methylation - 
the percent methylation at all CpG sites shared across all individuals. 
(C-F) Histone modifications - the peak heights of the indicated histone
modification for sites shared across all individuals.}
\label{fig:pc_plots}
\end{figure}

\begin{figure}[ht]
\centering
\caption{Overview of chromatin state composition, genomic distribution,
and effect on expression. The left most panel shows the emission
probabilites for each histone modification in each chromatin state. 
Blue indicates the absence of the histone modification, and red 
indicates the presence of the modification. The panel labeled
genomic enrichment shows the distribution of each state around 
functional elements in the genome. Red indicate that the
state is more likely to be found near the annotated functional 
element than expected by chance. Blue indicates that the state
is less likely to be found near the annotated functional element
than expected by chance. Abbreviations are as follows: TFBS = 
transcription factor binding sites, cCRE = candidate cis-regulatory
element [32728249], TSS = transcription start site, TES = transcription 
end site. The panel labeled Expression Effects shows the effect of the 
presence of each state on gene expression when it varies across strains. 
Bars are colored based on the size and direction the state's effect on
expression. These colors are used throughout the paper. For reference, 
we also show the overall effect of each state on gene expression across 
genes within hepatocytes (tan). The final column of the figure shows 
plausible annotations for each state based on combining the data in the 
previous three panels. The numbers in parentheses indicate the percent of 
the genome that was assigned to each state.}
\label{fig:state_overview}
\end{figure}

\begin{figure}[ht]
\centering
\caption{Relative abundance of chromatin states and methylated DNA. A. Each panel shows
the abundance of a single chromatin state relative to gene TSS and TES. The 
$y$-axis in each panel is the proportion of genes containing the state. Each
panel has an independent $y$-axis to better show the shape of each curve.
The $x$-axis is the relative gene position. The TSS and TES are marked as vertical
gray dashed lines. B. The same data shown in panel A, but with all states overlayed
onto a single $x$- and $y$-axis to show the relative abundance of the states. 
C. The density of CpG sites relative to the gene body. The $y$-axis shows the 
distance, in base pairs, to the next CpG site. This number goes down to almost
0 near the TSS showing that CpG sites are very densely packed in this region. 
CpG sites are less dense within the gene body than in the intergenic space. 
The blue polygon shows the 95\% confidence interval around the estimate. D. 
Percent methylation relative to the gene body. The $y$-axis shows the median 
percent methylation at CpG sites, and the $x$-axis shows relative gene position. 
CpG sites near the TSS are unmethylated relative to intragenic and intergenic
CpG sites.}
\label{fig:state_abundance}
\end{figure}

\begin{figure}[ht]
\centering
\caption{Effects of chromatin states on gene expression. Each column shows the effect
of each chromatin state on gene expression in a different context. The first
column shows the effect across genes in the inbred mice showing how chromatin
states are used within a single organism to increase the expression of some genes
and decrease the expression of other genes. The second column shows the effect
of chromatin state on gene expression across strains, showing how variation in
chromatin state across strains leads to variation in expression of individual
genes across strains. The third column shows the effect of imputed chromatin 
state on gene expression in a population of diversity outbred mice showing the
effect of variation in chromatin state across genetically diverse individuals
on local gene expression. Each column of panels is plotted on a single scale
for the $y$-axis so the magnitude of the effects in a single column can be 
compared directly to each other. Across a single row, the scale of the $y$-axis 
varies to highlight the similarity of the shape of each curve in each different 
setting. The final column shows the annotation of each state for comparison with
its effects on gene expression. All $y$-axes is the $\beta$ coefficient from the 
linear model shown in equation [REF]. All $x$-axes show the relative 
position along the gene body running from just upstream of the TSS to just downstream 
of the TES. Vertical gray dashed lines mark the TSS and TES in all panels.}
\label{fig:state_effects}
\end{figure}

\begin{figure}[ht]
\centering
\caption{Effect of DNA methylation on gene expression (A) across gene expression
in hepatocytes and (B) across inbred strains. Dark gray line shows estimate
of the effect of percent DNA methylation on gene expression. The $x$-axis is
normalized position along the gene body running from the transcription start
site (TSS) to the transcription end site (TES), marked with vertical gray dashed
lines. The horizontal solid black line indicates an effect of 0. 
The shaded gray area shows 95\% confidence interval arond the model fit.}
\label{fig:DNA_methylation_effect}
\end{figure}

\begin{figure}[ht]
\centering
\caption{Example of epigenetic states and imuptation results for a single 
gene, \textit{Pkd2}. (A) The variance in DO gene expression explained at 
each position along the gene body by each of the imputed genomic 
features: SNPs - red X's, Chromatin State - blue plus signs, and 
Percent Methylation - green circles. The horizontal dashed line shows 
the variance explained by the haplotype. For reference, the arrow 
below this panel runs from the TSS of \textit{Pkd2} to the TES and 
shows the direction of transcription. (B) The chromatin states assigned 
to each 200 bp window in this gene for each inbred mouse strain. States 
are colored by their effect on gene expression in the inbred mice. Red 
indicates a positive effect on gene expression, and blue indicates a 
negative effect. Each row shows the chromatin states for a single inbred 
strain, which is indicated by the label on the left. (C) SNPs along the 
gene body for each inbred strain. The reference genotype is shown in gray. 
SNPs are colored by genotype as shown in the legend. (D) Percent DNA 
methylation for each inbred strain along the \textit{Pkd2} gene body. 
Percentages are binned into 0\% (blue) 50\% (yellow) and 100\% (red). 
(E) Haplotype effects for expression of \textit{Pkd2} in the DO. 
Haplotype effects are colored by from which each allele was derived. 
(F) \textit{Pkd2} expression levels across inbred mouse strains. For 
ease of comparison, all panels B through F are shown in the same order 
as the haplotype effects.}
\label{fig:example_gene}
\end{figure}

\begin{figure}[ht]
\centering
\caption{Chromatin state explains variation in gene expression in an outbred 
population. A. Distributions of gene expression variance explained by different 
genomic features: local haplotype, local imputed chromatin state, local SNP 
genotype, and local imputed DNA methylation status. B. Direct comparisons of 
variance explained by local haplotype, and the three other genomic features: 
imputed chromatin state, SNP genotype, and imputed DNA methylation status. 
Blue lines show $y = x$. Each point is a single transcript.}
\label{fig:effect_distrubutions}
\end{figure}

\pagebreak

\hypertarget{supplemental-figure-legends}{%
\section{Supplemental Figure
Legends}\label{supplemental-figure-legends}}

\begin{figure}[ht]
\centering
\caption{Comparison of emissions probabilities across all ChromHMM models.
Each row contains data for a single ChromHMM model fit to the number of
states indicated on either side of the row. Each set of four columns shows
data for each of the four histone modifications. Each set is separated from
the next by a column of gray for ease of visualization. The bottom row, the
reference row, shows the ideal state that all model states are being compared
to. Blue indicates absence of the histone mark and red indicates presence.
For each ChromHMM model, each state was assigned to one of the reference states
using an emissions probability of 0.3 as a threshold for presence of the histone
modification. If a state was not present in the given model, the corresponding
area is shown in gray. Emissions probabilities near 0 are shown in blue, and 
probabilities near 1 are shown in red. Orange and yellow indicate intermediate
probabilities. Aligning the states across all models shows a remarkable 
stability in the emissions across models, seen as vertical bars of consistent 
color.}
\label{supp_fig:model_emissions_comparison}
\end{figure}

\begin{figure}[ht]
\centering
\caption{Comparison of state abundance across all ChromHMM models. The left-most column
shows the annotation for each state. Unannotated states are marked with a dash. The 
binary heatmap indicates which histone modifications were present in each state: 1 
indicates presence, and 0 indicates absence. The histone modifications are labeled
at the bottom of each column. The continuous heatmap shows the abundance of each state
(in rows) in each ChromHMM model (in columns). The abundance is the proportion of 
transcribed genes with the state present. Less abundant states are shaded blue, and 
more abundant states are shaded yellow, orange, and red. The number of states in the
model is indicated at the bottom of each column. The black box highlights the model
used in this study -- the 14-state model. State abundance was remarkably
stable across the different models.}
\label{supp_fig:model_abundance_comparison}
\end{figure}

\begin{figure}[ht]
\centering
\caption{Comparison of state effect across all ChromHMM models. This figure is 
identical to Figure \ref{supp_fig:model_abundance_comparison}, except that the
cells in the continuous heatmap show the effect of each state on gene expression
across all ChromHMM models. The effect was the $\beta$ coefficient derived from
a linear model. Similar to state abundance, the effects were remarkably stable
across models.}
\label{supp_fig:model_effect_comparison}
\end{figure}

\begin{figure}[ht]
\centering
\caption{Schematic for imputation of histone modifications into the 
DO mice. For a single transcript imputation was made by multiplying 
a three-dimensional array, containing chromatin state by strain by
position, by a two-dimensional array, contatining haplotype probabilities
by DO individual, to create a three-dimensional array, containing 
individual by position by chromatin state probability.}
\label{supp_fig:imputation}
\end{figure}

\hypertarget{references}{%
\section*{References}\label{references}}
\addcontentsline{toc}{section}{References}

\nolinenumbers


\end{document}
