% Template for PLoS
% Version 3.5 March 2018
%
% % % % % % % % % % % % % % % % % % % % % %
%
% -- IMPORTANT NOTE
%
% This template contains comments intended
% to minimize problems and delays during our production
% process. Please follow the template instructions
% whenever possible.
%
% % % % % % % % % % % % % % % % % % % % % % %
%
% Once your paper is accepted for publication,
% PLEASE REMOVE ALL TRACKED CHANGES in this file
% and leave only the final text of your manuscript.
% PLOS recommends the use of latexdiff to track changes during review, as this will help to maintain a clean tex file.
% Visit https://www.ctan.org/pkg/latexdiff?lang=en for info or contact us at latex@plos.org.
%
%
% There are no restrictions on package use within the LaTeX files except that
% no packages listed in the template may be deleted.
%
% Please do not include colors or graphics in the text.
%
% The manuscript LaTeX source should be contained within a single file (do not use \input, \externaldocument, or similar commands).
%
% % % % % % % % % % % % % % % % % % % % % % %
%
% -- FIGURES AND TABLES
%
% Please include tables/figure captions directly after the paragraph where they are first cited in the text.
%
% DO NOT INCLUDE GRAPHICS IN YOUR MANUSCRIPT
% - Figures should be uploaded separately from your manuscript file.
% - Figures generated using LaTeX should be extracted and removed from the PDF before submission.
% - Figures containing multiple panels/subfigures must be combined into one image file before submission.
% For figure citations, please use "Fig" instead of "Figure".
% See http://journals.plos.org/plosone/s/figures for PLOS figure guidelines.
%
% Tables should be cell-based and may not contain:
% - spacing/line breaks within cells to alter layout or alignment
% - do not nest tabular environments (no tabular environments within tabular environments)
% - no graphics or colored text (cell background color/shading OK)
% See http://journals.plos.org/plosone/s/tables for table guidelines.
%
% For tables that exceed the width of the text column, use the adjustwidth environment as illustrated in the example table in text below.
%
% % % % % % % % % % % % % % % % % % % % % % % %
%
% -- EQUATIONS, MATH SYMBOLS, SUBSCRIPTS, AND SUPERSCRIPTS
%
% IMPORTANT
% Below are a few tips to help format your equations and other special characters according to our specifications. For more tips to help reduce the possibility of formatting errors during conversion, please see our LaTeX guidelines at http://journals.plos.org/plosone/s/latex
%
% For inline equations, please be sure to include all portions of an equation in the math environment.
%
% Do not include text that is not math in the math environment.
%
% Please add line breaks to long display equations when possible in order to fit size of the column.
%
% For inline equations, please do not include punctuation (commas, etc) within the math environment unless this is part of the equation.
%
% When adding superscript or subscripts outside of brackets/braces, please group using {}.
%
% Do not use \cal for caligraphic font.  Instead, use \mathcal{}
%
% % % % % % % % % % % % % % % % % % % % % % % %
%
% Please contact latex@plos.org with any questions.
%
% % % % % % % % % % % % % % % % % % % % % % % %

\documentclass[10pt,letterpaper]{article}
\usepackage[top=0.85in,left=2.75in,footskip=0.75in]{geometry}

% amsmath and amssymb packages, useful for mathematical formulas and symbols
\usepackage{amsmath,amssymb}

% Use adjustwidth environment to exceed column width (see example table in text)
\usepackage{changepage}

% Use Unicode characters when possible
\usepackage[utf8x]{inputenc}

% textcomp package and marvosym package for additional characters
\usepackage{textcomp,marvosym}

% cite package, to clean up citations in the main text. Do not remove.
% \usepackage{cite}

% Use nameref to cite supporting information files (see Supporting Information section for more info)
\usepackage{nameref,hyperref}

% line numbers
\usepackage[right]{lineno}

% ligatures disabled
\usepackage{microtype}
\DisableLigatures[f]{encoding = *, family = * }

% color can be used to apply background shading to table cells only
\usepackage[table]{xcolor}

% array package and thick rules for tables
\usepackage{array}

% create "+" rule type for thick vertical lines
\newcolumntype{+}{!{\vrule width 2pt}}

% create \thickcline for thick horizontal lines of variable length
\newlength\savedwidth
\newcommand\thickcline[1]{%
  \noalign{\global\savedwidth\arrayrulewidth\global\arrayrulewidth 2pt}%
  \cline{#1}%
  \noalign{\vskip\arrayrulewidth}%
  \noalign{\global\arrayrulewidth\savedwidth}%
}

% \thickhline command for thick horizontal lines that span the table
\newcommand\thickhline{\noalign{\global\savedwidth\arrayrulewidth\global\arrayrulewidth 2pt}%
\hline
\noalign{\global\arrayrulewidth\savedwidth}}


% Remove comment for double spacing
%\usepackage{setspace}
%\doublespacing

% Text layout
\raggedright
\setlength{\parindent}{0.5cm}
\textwidth 5.25in
\textheight 8.75in

% Bold the 'Figure #' in the caption and separate it from the title/caption with a period
% Captions will be left justified
\usepackage[aboveskip=1pt,labelfont=bf,labelsep=period,justification=raggedright,singlelinecheck=off]{caption}
\renewcommand{\figurename}{Fig}

% Use the PLoS provided BiBTeX style
% \bibliographystyle{plos2015}

% Remove brackets from numbering in List of References
\makeatletter
\renewcommand{\@biblabel}[1]{\quad#1.}
\makeatother



% Header and Footer with logo
\usepackage{lastpage,fancyhdr,graphicx}
\usepackage{epstopdf}
%\pagestyle{myheadings}
\pagestyle{fancy}
\fancyhf{}
%\setlength{\headheight}{27.023pt}
%\lhead{\includegraphics[width=2.0in]{PLOS-submission.eps}}
\rfoot{\thepage/\pageref{LastPage}}
\renewcommand{\headrulewidth}{0pt}
\renewcommand{\footrule}{\hrule height 2pt \vspace{2mm}}
\fancyheadoffset[L]{2.25in}
\fancyfootoffset[L]{2.25in}
\lfoot{\today}

%% Include all macros below

\newcommand{\lorem}{{\bf LOREM}}
\newcommand{\ipsum}{{\bf IPSUM}}


% Pandoc citation processing




\usepackage{forarray}
\usepackage{xstring}
\newcommand{\getIndex}[2]{
  \ForEach{,}{\IfEq{#1}{\thislevelitem}{\number\thislevelcount\ExitForEach}{}}{#2}
}

\setcounter{secnumdepth}{0}

\newcommand{\getAff}[1]{
  \getIndex{#1}{The Jackson Laboratory}
}

\providecommand{\tightlist}{%
  \setlength{\itemsep}{0pt}\setlength{\parskip}{0pt}}

\begin{document}
\vspace*{0.2in}

% Title must be 250 characters or less.
\begin{flushleft}
{\Large
\textbf\newline{Correcting for relatedness in standard mouse mapping
populations; and something about
epistasis} % Please use "sentence case" for title and headings (capitalize only the first word in a title (or heading), the first word in a subtitle (or subheading), and any proper nouns).
}
\newline
% Insert author names, affiliations and corresponding author email (do not include titles, positions, or degrees).
\\
Catrina Spruce\textsuperscript{\getAff{The Jackson Laboratory}},
Anna L. Tyler\textsuperscript{\getAff{The Jackson Laboratory}},
Many more people\textsuperscript{\getAff{JAX-MG and JAX-GM}},
Gregory W. Carter\textsuperscript{\getAff{The Jackson
Laboratory}}\textsuperscript{*}\\
\bigskip
\textbf{\getAff{The Jackson Laboratory}}600 Main St.~Bar Harbor, ME,
04609\\
\bigskip
* Corresponding author: Gregory.Carter@jax.org\\
\end{flushleft}
% Please keep the abstract below 300 words
\section*{Abstract}
This abstract is in the yaml header. The easier-to-edit one is below.

% Please keep the Author Summary between 150 and 200 words
% Use first person. PLOS ONE authors please skip this step.
% Author Summary not valid for PLOS ONE submissions.
\section*{Author summary}
The author summary goes here

\linenumbers

% Use "Eq" instead of "Equation" for equation citations.
\hypertarget{abstract}{%
\section{Abstract}\label{abstract}}

It is well known that epigenetic modifications, such as histone
modifications, and DNA methylation are a major mode of regulating gene
transcription.

It is not well known how variation in epigenetic modifications across
genetically distinct individuals contributes to heritable variation in
gene expression.

When we map an eQTL, how much of the effect of the eQTL is mediated
through epigenetic modifications?

We investigated this question in genetically diverse mice.

local imputed histone modifications matched eQTL extremely well,
suggesting that a large portion of variation in gene expression mapped
to local genotype is mediated through histone modifications.

In contrast percent DNA methylation is not determined by local genetics,
and does not contribute to eQTLs.

\hypertarget{introduction}{%
\section{Introduction}\label{introduction}}

It is well established that epigenetic modifications, such as histone
modifications, and DNA methylation influence gene expression
{[}26704082, 22641018, 22781841{]}. Across cell types, unique
combinatorial patterns of histone modifications mark chromatin states
that establish cell type-specific patterns of gene expression
{[}20657582, 21441907{]}. Similarly, the methylation of CpG sites around
gene promoters and enhancers influences transcription in a cell
type-specific manner {[}21701563, 20720541{]}.

Cell type-specific patterns of histone modifications and DNA methylation
are established during development. The result is a canonical epigenetic
landscape for coordination of major patterns of gene expression for each
cell type {[}sources about development{]}. As an organism ages and
responds to its environment, patterns of both histone modifications
{[}citation{]} and of DNA methylation change {[}citation{]}. Such
changes have been linked to scenescence {[}Horvath clock{]} and cancer
{[}citations{]}.

Epigenetic modifications coordinate the usage of a single genome to be
used for many different types of cells with diverse morphology and
physiology. This amazing feature of epigenetic modifications has been
intensely studied, and the variation in epigenetic landscapes across
cell types has been extensively documented. Less well understood,
however, is the role that genetic variation plays in determining
epigenetic landscapes.

Across genetically diverse populations of humans or mice, individual
cell types, such as hepatocytes, or cardiomyocytes, have globally
similar gene expression profiles that define their role in the greater
organism. However, it is also true that across individuals, gene
expression varies widely within the global constraints of cell type.
This variation can increase or decrease an organism's risk of developing
disease. Variation in gene expression has been extensively mapped to
variation in genetic loci, or expression quantitative trait loci (eQTL).
Large, coordinated efforts, such as the Genotype-Tissue Expression
(GTEx) Project {[}32913073, 32913075{]} have identified and catalogued
many such loci in humans, and countlessindependent studies have
identified eQTL in mice and other model organisms.

Although the link between genetic variation and gene expression has been
well studied, there is relatively little known about inter-individual
variation in epigenetic modifications, and how these variations are
related to variations in genotype and gene expression. The generation of
a more complete picture of inter-individual variation in epigenetic
modifications has the potential to improve our understanding of the
mechanics of gene regulation, improve our understanding of how cell
type-specific epigenetic landscapes are established, and to improve the
functional annotation of the genome as it relates to the regulation of
gene expression.

Advances in chromatin immunoprecipitation (ChIP) and sequencing
technologies now enable genome-wide surveys of histone modifications
with relatively few cells {[}20077036{]}, thus opening the door to the

Such variation can be mapped to genetic variation

Estimates of the heritability

It is not well known how variation in epigenetic modifications across
genetically distinct individuals contributes to heritable variation in
gene expression.

Estimates of the heritability of epigenetic features range widely

Early in life, the landscape of epigenetic patterns is

When we map an eQTL, how much of the effect of the eQTL is mediated
through epigenetic modifications?

We investigated this question in genetically diverse mice.

We conducted a survey of four histone modifications known to be
correlated with gene transcription across nine inbred strains of mice.
We also surveyed DNA methylation in these strains.

We looked at how both histone modifications and DNA methylation were
associated with transcription variation across strains. We further
imputed epigenetic states in a population of diversity outbred mice to
more directly investigate the extent to which eQTLs are driven by
variation in epigenetic modifications

histone modifications, at least early in life, are determined by local
genotype.

DNA methylation is not determined genetically

GWAS hits tend to be in non-coding regions of the genome estimated that
most common disease variants work by altering gene expression rather
than protein function

These disease-associated SNPs likely fall into functional regions of the
genome

eQTLs - what are we measuring when we measure eQTL?

The identity of a hepatocyte is deterimined through patterns of gene
expression. Patterns of gene expression are determined in part through
patterns of genotype, DNA methylation, and chromatin modifications.

Within a given cell type, how do variations in local genetics and
epigenetics influence gene expression?

Across mouse strains, gene expression in hepatocytes is largely similar.
For the most part, genes that are highly expressed in one strain are
highly expressed in another. However, there are subtle variations in
gene expression that are based on strain background.

This variation in gene expression across strains is related to genetic
and epigenetic factors. Here we explore how local genotype, chromatin
modifications, and DNA methylation influence strain differences in gene
expression.

patterns of chromatin state in hepatocytes varied across strains
patterns of DNA methylation in hepatocytes varied across strains
patterns of gene expression in hepatocytes varied across strains

major axes of variation were similar in all cases, i.e.~PWK and CAST
were most divergent, while other strains clustered together

Each of these epigenetic-expression patterns represent functioning
hepatocytes these are ``good enough'' solutions to make hepatocytes
{[}22859671{]}

There is evidence that, especially early in life, chromatin
modifications are genetically determined {[}cite{]}.

\hypertarget{materials-and-methods}{%
\section{Materials and Methods}\label{materials-and-methods}}

\hypertarget{inbred-mice}{%
\subsection{Inbred Mice}\label{inbred-mice}}

information about housing, animal use, etc.

\hypertarget{hepatocyte-acquisition}{%
\subsubsection{Hepatocyte acquisition}\label{hepatocyte-acquisition}}

Samples were taken from 12-week female mice of nine inbred mouse
strains: 129S1/SvImJ, A/J, C57BL/6J, CAST/EiJ, DBA/2J NOD/ShiLtJ,
NZO/HlLtJ, PWK/PhJ, and WSB/EiJ. Eight of these strains are the eight
strains that served as founders of the Collaborative Cross/Diversity
Outbred mice {[}REF{]}. The ninth strain, DBA/2J, will facilitate the
interpretation of existing and forthcoming genetic mapping data obtained
from the BxD recombinant inbred strain panel {[}REF{]}. Mice were aged
and processed in groups to maintain a steady sample preparation
workflow. Mice were housed, born, and aged in the same mouse room, with
uniformity in timing, diet, and all other possible conditions. Female
mice were used for all experiments due to potentially confounding
effects from variation in testosterone among males that can affect liver
gene expression, as well general experience that female expression is
less variable than male in multiple tissues. This will also facilitate
the analysis of maternal effects on offspring in later studies. Three
mice were used from each strain.

\hypertarget{liver-perfusion}{%
\subsubsection{Liver perfusion}\label{liver-perfusion}}

To purify hepatocytes from the liver cell population, the mouse livers
were perfused with collagenase to digest the liver into a single-cell
suspension, and then isolated using centrifugation. Mice were harvested
at 9:00 AM and sacrificed by cervical dislocation. Mice were placed over
a stack of paper towels in preparation to catch excess liquid, and the
appendages were pinned out to hold the body in place. to keep the fur
from contaminating the liver sample later, the fur was wiped down with
70\% ethanol. The mouse skin was then cut open and peeled back to the
appendages to allow clear access to the abdominal cavity. The fascia was
cut open and back to the ribs, being careful to not nick the liver.
Moving the intestines and stomach to the right side, the vena cava and
hepatic portal vein should be clearly visible below the liver.

For the perfusion, a 23G x \(\frac{3}{4}\)'\,' BD Vacutainer Safety-Lok
needle (REF 367297) was attached to 1.6mm ID BioRad Tygon tubing
(R-3603) connected to a Pharmacia peristaltic pump that allows a flow of
up to 8 ml/min. The liver will be processed with three solutions: 5mM
EGTA in Leffert's buffer, Leffert's buffer wash, and 87 CDU/mL Liberase
collagenase with 0.02\% CaCl2 in Leffert's buffer. The three solutions
were at 37\(^{\circ}\)C before perfusion.

The needle was placed into the vena cava for the perfusion superior to
the kidneys and inferior to the liver. With the peristaltic pump running
slowly, the vena cava was pierced at shallow 15\(^{\circ}\) angle and
the needle was inserted to a shallow depth (around 2-3mm of the needle
tip in the vena cava). Once the needle is inserted into the vena cava,
the volume on the peristaltic pump is increased to 5-7mL/min. The liver
will immediately blanch, and the hepatic portal vein is immediately
severed to allow flushing of the liver.

The 1x EGTA buffer was used to flush the blood out of the liver and
start the digestion of the desmosomes connecting the liver cells. To
help with the perfusion, pressure was applied to the hepatic portal vein
for 5 second intervals causing more solution to be forced through the
liver, which can be seen visually by the liver swelling. After 35ml of
the 1x EGTA solution is passed through the liver, the solution was
switched to the 1x Leffert's buffer. The pump was turned off during the
switch to prevent air from being sucked into the tubing while the tubing
is transferred to the new solution. To wash, 7-10ml of the Leffert's
buffer was passed through the liver to flush out the EGTA, which
otherwise chelates the calcium ions necessary for collagenase activity
in the next buffer. The pump was turned off again to switch to the
Liberase solution. To digest the liver, 25-50mL of Liberase solution
(\(\sim4.3\) wunsch units) was passed through the liver. Throughout the
perfusion process, periodic pressure was applied to the hepatic portal
vein to help pump the buffers more completely through the liver. As the
liver was digested with the Liberase, it will swell and look soggy and
limp. Over-digestion leads to increased contamination with
non-hepatocyte cell types, and further reduces cell viability.

After perfusion, which takes around 15-20min to complete, the liver was
carefully cut out of the abdominal cavity and placed in a petri dish
with 35 mL ice-cold Leffert's buffer with 0.02\% CaCl\(_{2}\). The
digested liver was passed through Nitex 80 \(\mu\)m nylon mesh (cat
\#03-80/37) into a 50mL conical, using additional ice-cold Leffert's
buffer with 0.02\% CaCl2 if necessary, and a rubber policeman. After the
liver cells from both animals were collected, they were put through two
wash and spin cycles to purify the hepatocytes and remove other types of
cells. To isolate the hepatocytes, the much larger size of the
hepatocyte cells was exploited in very slow 4 min, 50 x g spins that
leave smaller other cell types in suspension. After each spin, the
solution was decanted as waste, and the enriched cell pellet of
hepatocytes was resuspended in 30ml ice-cold Leffert's buffer with
0.02\% CaCl\(_{2}\). After the second spin, the solution should be
almost clear, indicating that other cell types have been removed. The
hepatocytes are resuspended in room temperature PBS, counted, and volume
adjusted to \(1 x 10^{6}\) cells/mL.

We aliquoted \(5 x 10^{6}\) cells for each RNA-Seq and bisulfite
sequencing, and the rest were cross-linked for ChIP assays. Two
\(5 x 10^6\) aliquots (5mLs) of liver cells were removed into two 15mL
conicals. These were spun down at 200 rpm for 5 min, and resuspended in
\(1200\mu L\) RTL+BME (for RNA-Seq) or frozen as a cell pellet in liquid
nitrogen (for bisulfite sequencing). Meanwhile, 37\% formaldehyde in
methanol (VENDOR) were added to the remaining cells to a final
concentration of 1\%. The cells were rotated at room temperature for 5
min to cross-link protein complexes to the DNA bound to them. After
cross-linking, 10x glycine (VENDOR) is added to a final concentration of
125 mM and rotated for 5 min to quench the formaldehyde and stop
cross-linking. The cells were spun down at 2000 rpm for 5 min, decanted,
and resuspended in PBS to \(5 x 10^6\) cells/mL. The cells were divided
into \(5x10^6\) aliquots in 2mL tubes. The tubes were spun down again at
5000 x g for 5 min, decanted, and the cell pellets frozen in liquid
nitrogen. All cell samples were stored at -80°C until used.

\hypertarget{hepatocyte-histone-binding-and-gene-expression-assays}{%
\subsubsection{Hepatocyte histone binding and gene expression
assays}\label{hepatocyte-histone-binding-and-gene-expression-assays}}

Hepatocyte samples from 30 treatment and control mice were used in the
following assays:

\begin{enumerate}
\def\labelenumi{\arabic{enumi}.}
\tightlist
\item
  RNA-seq to quantify mRNA and long non-coding RNA expression, with
  approximately 30 million reads per sample.
\item
  Reduced-representation bisulfate sequencing to identify methylation
  states of approximately two million CpG sites in the genome. The
  average read depth is 20-30x.
\item
  Chromatin immunoprecipitation and sequencing to assess binding of the
  following histone marks:
\end{enumerate}

\begin{enumerate}
\def\labelenumi{\alph{enumi}.}
\tightlist
\item
  H3K4me3 to map active promoters
\item
  H3K4me1 to identify active and poised enhancers
\item
  H3K27me3 to identify closed chromatin
\item
  H3K27ac, to identify actively used enhancers
\item
  A negative control (input chromatin) Samples are sequenced with
  \(\sim40\) million reads per sample.
\end{enumerate}

The samples for RNA-Seq in RTL+BME buffer were sent to The Jackson Lab
Gene Expression Service for RNA extraction and library synthesis.

\hypertarget{histone-chromatin-immunoprecipitation-assays}{%
\subsubsection{Histone chromatin immunoprecipitation
assays}\label{histone-chromatin-immunoprecipitation-assays}}

The H3K4me1 and H3K4me3 histone chromatin immunoprecipitation assays
were performed on cross-linked hepatocytes using similar protocols. For
all histone ChIP assays, the crosslinked chromatin was prepared the same
way. First, the aliquot of \(5x10^6\) hepatocyte cells was lysed to
release the nuclei by rotating the sample in hypotonic buffer for 20 min
at \(4^\circ\)C. The cells were pelleted by spinning for 10min, 10K x G,
at \(4^\circ\)C. The cells were resuspended in 130ul MNase buffer with
1mM PMSF (VENDOR) and 1x protease inhibitor cocktail (Roche VENDOR) to
prevent histone protein degradation, then digested with 15U of MNase.
The micrococcal nuclease digests the exposed DNA, but leaves the
nucleosome-bound DNA intact. After 10min of incubation at \(37^\circ\)C,
the chromatin was digested into primarily mononucleosomes. This was
confirmed by DNA-purification of the MNase-digested chromatin run out on
an agarose gel, which yielded mostly 150bp fragments, and few 300bp
fragments. The MNase digestion was stopped by adding EDTA to 10mM, and
incubating on ice for 5 min. The digested chromatin was purified by
spinning out insoluble parts at top speed for 10 min at \(4^\circ\)C.
The chromatin was transferred to a new tube and spun again to further
remove impurities and reduce background in the ChIP assays. The final
chromatin was transferred to a fresh tube, and used immediately in the
ChIP.

To prepare for the ChIP, \(20\mu L/1x10^6\) cells Dynabead Protein G
beads were aliquoted into an Eppendorf tube. A magnetic tube holder was
used to attract the beads to the wall of the tube, and then the solution
was carefully pipetted off, leaving only the beads behind. The beads
were washed twice with buffer to prepare them for binding to the
antibody. For this binding step and the chromatin binding step, the
buffer used was either RIPA buffer for the H3K4me3 and K3K27me3 ChIPs,
or ChIP buffer (VENDOR) for the H3K4me1 ChIP. The ChIP buffer was
gentler and less stringent than RIPA buffer, which was better for the
weaker binding of the H3K4me1 antibody that was used. The buffers were
supplemented with 50 mg/mL BSA (VENDOR) and 0.5 mg/mL Herring Sperm DNA,
both of which are blocking agents that reduce background and
non-specific binding. The ChIP assays also varied in the amount of input
chromatin and corresponding size of the reaction that was necessary to
yield sufficient DNA for sequencing. H3K4me3 ChIP needed only
\(1.5 x 10^6\) cells, and H3K4me1 and K3K27me3 ChIP used \(4 x 10^6\)
cells. To perform the ChIP, \(20\mu\)L of Dynabeads per \(1 x 10^6\)
cells is incubated with \(5\mu L\) of histone antibody for \(>20\)min in
\(50\mu L/1x10^6\) cells RIPA (or ChIP) buffer supplemented with 50
mg/mL BSA, 0.5 mg/mL Herring Sperm DNA, 1xPIC, and 1mM PMSF. The
antibodies used were (XXX). Once the antibody was bound to the
Dynabeads, the beads were washed twice with \(100\mu L/1x10^6\) cells
RIPA buffer with BSA and Herring Sperm DNA.

Next, the MNase-digested chromatin were added, which was at a
concentration of \(1x10^6\) cells/\(25\mu L\). The ChIP reaction was
incubated overnight with rotation at \(4^\circ\)C, to allow the histone
protein to bind to the antibody, which was bound to the magnetic beads.
In order to calculate enrichment for each ChIP sample, a known amount
(\(10\) or \(20\mu L\)) of MNase-digested input chromatin was saved.\\
The next morning, the ChIPs underwent a series of washes to remove
unbound chromatin. The H3K4me3 and H3K27me3 ChIPs were washed 3x with
\(100\mu L/1x10^6\) cells RIPA buffer, and the H3K4me1 ChIP was washed
with a low salt wash (0.1\% SDS, 1\% Triton X-100, 2mM EDTA, 20mM
Tris-HCl pH 8, 150 mM NaCl), a high salt wash (0.1\% SDS, 2\% Triton
X-100, 2mM EDTA, 20mM, Tris-HCl, pH 8, 500mM NaCl), and a LiCl wash
(0.25 MLiCl, 1\% IGEPAL-CA630, 1\% deoxycholic acid (sodium salt), 1 mM
EDTA, 10 mM Tris-HCl pH 8). After three washes, the ChIPs were washed
twice with TE buffer and transferred to a new tube during the last TE
wash to reduce background. At this point, the histone of interest and
the histone-bound DNA fragment had been purified from the
MNase-digested, cross-linked chromatin, and was bound by
histone-specific antibody to the magnetic Dynabeads. In the next step, a
high-salt elution buffer is used to degrade the antibody binding
interactions to the beads and the histone, and concurrently, proteinase
K is added to digest the protein away from the DNA-protein complexes.
The ChIP was incubated with the elution buffer and proteinase K at
\(68^\circ\)C for \(>6\) hours to liberate the DNA. At the same time,
the saved input chromatin was also digested in the same buffer.
Afterwards, the beads were removed using the magnet, and the DNA was
purified using the Qiagen PCR purification kit. Quantification was
performed using the Qubit quantification system, which is accurate to
\(0.02 ng/\mu L\) and only requires a small amount of sample to measure
concentration. The ChIP sample was enriched for only DNA that was bound
to the histone of interest. The goal for each ChIP was to yield 10 ng of
ChIP DNA for sequencing. Not all samples met this criterion, and the
H3K4me1 ChIPs often had a total yield of \(\sim 2 ng\) of DNA.

To test the efficiency of the ChIPs, quantitative PCR using QuantiFAST
was performed. Two sets of primers were used, one set in a known region
of histone binding (positive control), and one set in a region without
histone binding (negative control). The qPCR was performed both on the
ChIP DNA and the input DNA. Then the relative enrichment of positive vs
negative assays was compared between the ChIP and input DNA.

The ChIP DNA was submitted to The Jackson Lab GES service for library
preparation and sequencing. Libraries were made using the Kapa Hyper
Prep kit with adapters at \(0.6\mu M\). The libraries were amplified by
10 cycles of PCR. These libraries were not size selected, although most
fragments were \(\sim150\) bp due to MNase-digestion. The samples were
sequenced with 40 or more million reads per sample, which is almost 2x
more reads than the ENCODE project, which sequenced using 20 million
reads.

\hypertarget{diversity-outbred-mice}{%
\subsection{Diversity Outbred mice}\label{diversity-outbred-mice}}

We used previously published data from a population of diversity outbred
(DO) mice {[}Svenson et al.~2012{]} to compare to the data collected
from the inbred mice. The DO population included males and females from
DO generations four through 11. Mice were randomly assigned to either a
chow diet (6\% fat by weight, LabDiet 5K52, LabDiet, Scott Distributing,
Hudson, NH), or a high-fat, high-sucrose (HF/HS) diet (45\% fat, 40\%
carbohydrates, and 15\% protein) (Envigo Teklad TD.08811, Envigo,
Madison, WI). Mice were maintained on this diet for 26 weeks.

\hypertarget{genotyping}{%
\subsubsection{Genotyping}\label{genotyping}}

All DO mice were genotyped as described in Svenson et
al.\textasciitilde(2012) using the Mouse Universal Genotyping Array
(MUGA) (7854 markers), and the MegaMUGA (77,642 markers) (GeneSeek,
Lincoln, NE). All animal procedures were approved by the Animal Care and
Use Committee at The Jackson Laboratory (Animal Use Summary \# 06006).

Founder haplotypes were inferred from SNPs using a Hidden Markov Model
as described in Gatti\textsubscript{\textit{et~al.}}2014. The MUGA and
MegaMUGA arrays were merged to create a final set of evenly spaced
64,000 interpolated markers.

\hypertarget{tissue-collection-and-gene-expression}{%
\subsubsection{Tissue collection and gene
expression}\label{tissue-collection-and-gene-expression}}

At sacrifice, whole livers were collected and gene expression was
measured using RNA-Seq as described in (Chick, Munger et
al.\textasciitilde2016, and Tyler et al.\textasciitilde2017). Transcript
sequences were aligned to strain-specific genomes, and we used an
expectation maximization algorithm (EMASE) to estimate read counts
(\url{https://github.com/churchill-lab/emase}).

\hypertarget{data-processing}{%
\subsection{Data Processing}\label{data-processing}}

\hypertarget{sequencing}{%
\subsubsection{Sequencing}\label{sequencing}}

The raw sequencing data from both RNA-Seq and ChIP-Seq was put through
the quality control program FastQC. FastQC identifies problems or biases
in either the sequencer run or the starting library material. The FastQC
readout includes total number of reads, sequence quality, duplication
level, and overrepresented sequences. All of our samples had comparable
quality levels and no outstanding flags. However, the ChIP-Seq data was
flagged for having a high level of duplicate reads. This can be
explained by the use of MNase to shear the DNA into 150 bp fragments. If
the binding positions of nucleosomes are fixed, then the MNase enzyme
will cleave the DNA in the same place in multiple cells, resulting in
duplicate pieces of DNA. Despite evidence that the duplication rate has
a biological explanation, duplicates were removed before downstream
analysis, as is typical in sequencing workflows, to avoid potential
biases caused by starting libraries that have less diversity.

For the sequence analysis, reads from each sample were mapped to
strain-specific pseudogenomes that integrate known SNPs from each
strain. While the B6 samples were aligned directly to the reference
mouse genome, the other samples were from genetically different strains.
Strain-specific sequence variation in transcripts can affect alignment
quality and result in biased estimates of abundance. To counteract
potential strain biases, sequencing data from each strain were aligned
to a custom strain pseudogenome, allowing a more precise
characterization of gene expression and histone binding. The
pseudogenomes were created using the EMASE computational program
{[}REF{]} designed to construct customized genomes based on known SNP
and indel attributes. The resulting custom genomes are called
pseudogenomes, because they are based on inserting small known
variations into the reference genome, but do not attempt whole genome
sequencing for each strain and complete rebuild the entire genomic
sequence from the scaffold up. The strain-specific pseudogenomes were
then used in the Bowtie mapping algorithm to align and map reads from
the RNA-Seq and ChIP-Seq experiments.

\hypertarget{quantifying-gene-expression}{%
\subsection{Quantifying gene
expression}\label{quantifying-gene-expression}}

Once the sequencing data was mapped to the custom genomes, edgeR is used
to quantify transcripts. The edgeR program uses a Trimmed Mean of
M-values (TMM), which adjusts each sample for library size and RNA
composition using the assumption that most genes are not differentially
expressed. The output is sample read count for each of the ENSMUSG
transcript ID's. Next, transcripts with less than 1 CPM in two or more
replicates were filtered to remove lowly expressed genes. Also, the data
were trimmed to include only protein-coding transcripts.

\hypertarget{chip-seq-quantification}{%
\subsubsection{ChIP-Seq quantification:}\label{chip-seq-quantification}}

After the ChIP-Seq sequencing data were mapped to the custom
pseudogenomes, peaks were called in each sample using MACS 1.4.2
{[}REF{]}, with a significance threshold of \(p \leq 10^{-5}\). In order
to compare peaks across strains, the MACS output peak coordinates were
converted to common B6 coordinates using g2g tools {[}REF{]}.

Annat's stuff to get fastq files to bam files bam to bed binarize bed
files

\hypertarget{quantifying-dna-methylation}{%
\subsection{Quantifying DNA
methylation}\label{quantifying-dna-methylation}}

Annat's stuff to get bed files.

\hypertarget{analysis}{%
\subsection{Analysis}\label{analysis}}

\hypertarget{filtering-transcripts}{%
\subsection{Filtering transcripts}\label{filtering-transcripts}}

For all gene expression data, we remove transcripts with extremely low
read counts, by filtering out those whose mean read count across all
individuals was less than five.

We used the R package sva {[}REF{]} to perform a variance stabilizing
transformation (vst) on the RNA-Seq read counts from both inbred and
outbred mice. In the inbred mice we used a blind transformation, while
in the outbred mice, we included DO wave and sex in the model. For eQTL
mapping, we performed rank Z normalization on the RNA-Seq read counts
across transcripts from the outbred mice.

\hypertarget{analysis-of-histone-modifications}{%
\subsection{Analysis of histone
modifications}\label{analysis-of-histone-modifications}}

\hypertarget{identification-of-chromatin-states}{%
\subsubsection{Identification of chromatin
states}\label{identification-of-chromatin-states}}

We used ChromHMM {[}29120462{]} to identify \emph{chromatin states},
which are unique combinations of the four chromatin modifications, for
example, the presence of both H3K4me3 and H3K4me1, but the absence of
the other two modifications. We conducted all subsequent analyses at the
level of the chromatin state.

To ensure we were analyzing the most biologically meaningful chromatin
states, we calculated chromatin states for all numbers of states between
four and 16, which is the maximum number of states possible with four
binary chromatin modifications (\(2^n\)). We then investigated a number
of features of each state in each model: presence/absence of histone
modifications, distribution patterns across the genome, and the effect
of each state on gene expression. We compared chromatin states from the
different models based on these analyses and selected the 14-state
model. Each of these analyses, and the model comparison, are described
below.

\hypertarget{emission-probabilities}{%
\subsubsection{Emission probabilities}\label{emission-probabilities}}

Emission probabilites are a primary output of ChromHMM (Figure XXXA).
They define the probability that each histone mark is present in each
detected state. Low probabilities suggest absence, or low levels of the
mark, and high probabilities suggest presence. To compare states to each
other and to annotate states, we declared a histone mark to be present
in a state if its emission probability was 0.3 or higher.

\hypertarget{genome-distribution-of-chromatin-states}{%
\subsubsection{Genome distribution of chromatin
states}\label{genome-distribution-of-chromatin-states}}

We investigated genomic distributions of chromatin states in two ways.
First, we used the ChromHMM function OverlapEnrichment to calculate
enrichment of each state around known functional elements in the mouse
genome. We analyzed the following features:

\begin{itemize}
\tightlist
\item
  \textbf{Transcription start sites (TSS)} - Annotations of TSS in the
  mouse genome were provided by RefSeq {[}26553804{]} and included with
  the release of ChromHMM, which we downloaded on December 9, 2019
  {[}29120462{]}.
\item
  \textbf{Transcription end sites (TES)} - Annotations of TES in the
  mouse genome were provided by RefSeq and included with the release of
  ChromHMM.
\item
  \textbf{Transcription factor binding sites (TFBS)} - We downloaded
  TFBS coordinates from OregAnno {[}26578589{]} using the UCSC genome
  browser {[}12045153{]} on May 4, 2021.
\item
  \textbf{Promoters} - We downloaded promoter coordinates provided by
  the eukaryotic promoter database {[}27899657,25378343{]}, through the
  UCSC genome browser on April 26, 2021.
\item
  \textbf{Enhancers} - We downloaded annotated enhancers provided by
  ChromHMM through the UCSC genome browser on April 26, 2021.
\item
  \textbf{Candidates of cis regulatory elements in the mouse genome
  (cCREs)} - We downloaded cCRE annotations provided by ENCODE
  {[}22955616{]} through the UCSC genome browser on April 26, 2021.
\item
  \textbf{CpG Islands} - Annotations of CpG islands in the mouse genome
  were included with the release of ChromHMM.
\end{itemize}

In addition to these enrichments around individual elements, we also
calculated chromatin state abundance relative to the main anatomical
features of a gene. For each transcribed gene, we generated a chromatin
state matrix with genomic position in rows, and mouse strains in
columns. Each cell contained the chromatin state assignment for a 200
base pair (bp) window, defined by ChromHMM, for each strain. We
normalized these bp positions for each gene, such that they ran from 0
at the transcription start site (TSS) to 1 at the transcription end site
(TES). We also included 1000 bp upstream of the TSS and 1000 bp
downstream of the TES, which were converted to values below 0 and above
1 respectively.

To normalize the coordinates, we first centered all coordinates on the
TSS of the gene by subtracting off the base pair position of the TSS.
Centered positions were then divided by the length of the gene in base
pairs from the TSS to the TES. We then binned the relative positions
into 41 bins defined by the sequence from -2 to 2 incremented by 0.1. If
a bin encompassed multiple positions in the gene, we assigned the mean
value of the feature of interest to the bin. To avoid potential
contamination from regulatory regions of nearby genes, we only included
genes that were at least 2kb from their nearest neighbor, for a final
set of 14048 genes.

\hypertarget{chromatin-state-and-gene-expression}{%
\subsubsection{Chromatin state and gene
expression}\label{chromatin-state-and-gene-expression}}

We calculated the effect of each chromatin state on gene expression. We
did this both across genes and across strains. The first analysis
identifies states that are associated with high expression and low
expression within the hepatocytes, and the second analysis investigates
whether variation in chromatin state across strains contributes to
variation in gene expression across strains.

For each transcribed gene, we calculated the proportion of the gene body
that was assigned to each chromatin state. We then fit a linear model
separately for each state to calculate the effect of state proportion
with gene expression:

\begin{equation*}{\label{eqn:chromatin_effect}}
y_{e} = \beta x_{s} + \epsilon
\end{equation*}

where \(y_{e}\) is the rank Z normalized gene expression of the full
transcriptome in a single inbred strain, and \(x_{s}\) is the rank Z
normalized proportion of each gene that was assigned to state \(s\). We
fit this model for each strain and each state to yield one \(\beta\)
coefficient with 95\% confidence interval. The effects were not
different across strains, so we averaged the effects and confidence
intervals across strains to yield one summary effect for each state.

To calculate the effect of each chromatin state across strains, we first
standardized transcript abundance across strains for each transcript. We
also standardized the proportion of each chromatin state for each gene
across strains. We then fit the same linear model, where \(y_{e}\) was a
rank Z normalized vector concatenating all standardized expression
levels across all strains, and \(x_{s}\) was a rank Z normalized vector
concatenating all standardized state proportions across all strains. We
fit the model for each state independently yielding a \(\beta\)
coefficient and 95\% confidence interval for each state.

In addition to calculating the effect of state proportion across the
full gene body, we also performed the same calculations in a
position-based manner. This second analysis yielded an effect of each
state at multiple points along the gene body and a more nuanced view of
the effect of each state.

\hypertarget{selecting-the-most-biologically-meaningful-model}{%
\subsubsection{Selecting the most biologically meaningful
model}\label{selecting-the-most-biologically-meaningful-model}}

We performed the above analyses on all states from the four-state model
to the 16-state model to find the most meaningful clustering of histone
modifications. Across all models, the states were remarkably stable
(Supplemental Figure XXX). As we increased the number of states detected
by the model, new states appeared, but previously detected states were
not disrupted. This stability was apparent in all state measures:
emissions probability patterns, overall abundance, effect on
expresssion, and localization along the genome. The one exception to
this stability was that highly abundant state (present in 65\% of
transcribed genes) detected first in the four-state model was split into
two distinct states in the 10-state model. These states were also highly
abundant (appearing in 40\% and 41\% of transcribed genes), and had
distinct genomic distributions and emissions probabilities (Supplemental
Figure XXX). These two states remained stable with increasing numbers of
clusters through to the 16-state model. States arising after the
10-state model were of lower abundance, appearing in 2\% or less of
transcribed genes.

All of the higher abundance states were established in the 10-state
model. However, as we moved toward higher numbers of clusters, the
resolution on the lower-abundance states improved in terms of the
emission probability profiles, and strength of the correlation with gene
expression. For example, the 14-state model better resolved a state that
had appeared in the 10-state model but was not strongly correlated with
gene expression. In the 14-state model, the emission patterns were
closer to binary, and the strength of the correlation with expression
was increased. Beyond 14 clusters, the new states identified were
extremely rare (1\% of transcripts or less), and were not strongly
correlated with gene expression. We thus selected the 14-state model and
the model with the most biologically meaningful clusters.

\hypertarget{analysis-of-dna-methylation}{%
\subsection{Analysis of DNA
methylation}\label{analysis-of-dna-methylation}}

\hypertarget{creation-of-dna-methylome}{%
\subsubsection{Creation of DNA
methylome}\label{creation-of-dna-methylome}}

We combined the DNA methylation data into a single methylome cataloging
the methylated sites across all strains. For each site, we averaged the
percent methylation across the three replicates in each strain. The
final methylome contained 5,311,670 unique sites across the genome.
Because methylated CpG sites can be fully methylated, unmethylated, or
hemi-methylated, we rounded the average percent methylation at each site
to the nearest 0, 50, or 100.

\hypertarget{distribution-of-cpg-sites}{%
\subsubsection{Distribution of CpG
sites}\label{distribution-of-cpg-sites}}

We used the enrichment function in ChromHMM described above to identify
enrichment of CpG sites around functional elements in the mouse genome.
We further performed a gene-based analysis of abundance similar to that
in the chromatin states. As a function of relative position on the gene
body, we calculated the density of CpG sites as the average distance to
the next downstream CpG site, as well as the percent methylation at each
site.

\hypertarget{effects-of-dna-methylation-on-gene-expression}{%
\subsubsection{Effects of DNA methylation on gene
expression}\label{effects-of-dna-methylation-on-gene-expression}}

As with chromatin state, we assessed the effect of DNA methylation on
gene expression both within strains (across genes), and across strains.
We used the same linear model described above, except that \(y_{s}\)
became the rank Z normalized percent methylation either across genes or
across strains. However, unlike with the chromatin states, we only
calculated the effects of DNA methylation on gene expression in a
position-dependent manner.

\hypertarget{imputation-of-genomic-features-in-diversity-outbred-mice}{%
\subsection{Imputation of genomic features in Diversity Outbred
mice}\label{imputation-of-genomic-features-in-diversity-outbred-mice}}

To assess the extent to which chromatin state and DNA methylation are
responsible for local expression QTLs, we imputed local chromatin state
and DNA methylation into a population of diversity outbred (DO) mice
described above and in Svenson et al.~2012. We compared the effect of
the imputed epigenetic features to imputed SNPs.

All imputations followed the same basic procedure: For each transcript,
we identified the haplotype probabilities in the DO mice at the genetic
marker nearest the gene transcription start site. This matrix held DO
individuals in rows and DO founder haplotypes in columns.

For each transcript, we also generated a three-dimensional array
representing the genomic features derived from the DO founders. This
array held DO founders in rows, feature state in columns, and genomic
position in the third dimension. The feature state for chromatin
consisted of states one through 14, for SNPs feature state consisted of
the genotypes A,C,G, and T (Fig XXX?).

We then multiplied the haplotype probabilities by each genomic feature
array to obtain the imputed genomic feature for each DO mouse. This
final array held DO individuals in rows, the genomic feature in the
second dimension, and genomic position in the third dimension. This
array is analagous to the genoprobs object in R/qtl2 (CITE). The genomic
position dimension included all positions from 1 kb upstream of the TSS
to 1 kb downstream of the TES. SNP data for the DO founders in mm10
coordinates were downloaded from the Sanger SNP database {[}1921910,
21921916{]}, on July 6, 2021.

To calculate the effect of each imputed genomic feature on gene
expression in the DO population, we fit a linear model. From this linear
model, we calculated the variance explained (\(R^2\)) by each genomic
feature, thereby relating gene expression in the DO to each position of
the imputed feature in and around the gene body.

\hypertarget{results}{%
\section{Results}\label{results}}

Gene expression varies widely and reproducibly across inbred strains of
mice. This is seen as a clustering of individuals from the same strain
in a principal component plot of the hepatocyte transcriptome across
strains (Figure XXX). The effect of genotype on this variation can be
measured in a mapping population as expression quantitative trait loci
(eQTL), which associate genetic variation with variation in gene
expression. In this study we investigated the extent to which variation
in epigenetic modifications, such as histone modifications and DNA
methylation, are associated with genetically controlled variation in
gene expression.

\hypertarget{chromatin-state-overview}{%
\subsection{Chromatin state overview}\label{chromatin-state-overview}}

To investigate this association, we used ChromHMM to identify 14
chromatin states composed of unique combinations of four histone
modifications in the hepatocytes of nine inbred strains of mice. Ppanel
A in Figure XXX shows the representation of each histone modification
across the states.

The states were distributed non-randomly around known functional
elements in the mouse genome (Figure XXXB). The majority of the states
were enriched around the TSS, and other TSS-related functional elements,
such as promoters and CpG islands. Two states (states 2 and 1) were
primarily found in intergenic regions, three states (states 9, 13, and
11) were enriched around known enhancers, and one (state 6) was enriched
predominantly near the TES. The majority of these states were also
associated with variation in gene expression (Figure XXXC). The colored
bars in this panel show the effect of each state on gene expression
variation across the inbred strains. For reference, the paired tan bars
show the effect of each chromatin state on gene expression in
hepatocytes. These effects are the same sign as the across-strain
effects, for the most part, and tend to be stronger.

By merging histone modification patterns with enrichments near
functional elements and the effects of each state on gene expression, we
were able to suggest annotations for many of the 14 chromatin states
(Figure XXXD). A more detailed description of these annotations is in
Table XXX.

The states in Figure XXX are shown in order of their effect on
expression, which helps illustrate several patterns in the data. The
state with the most negative effect on gene expression, state 1, is the
absence of all measured modifications. The next few states all contain
the repressive mark H3K27me2, and are all associated with reduced gene
expression. The states with the most positive effects on expression all
have some combination of the activating marks, H3K4me3, H3K4me1, and
H3K27ac, and the repressive mark is less commonly seen in these states
with positive effects. Maybe unsurprisingly, we were also better able to
annotate the states with strong negative effects and strong positive
effects, but the states in the middle with weak effects were more
difficult to annotate.

That states 1 and 2 were associated with reduced gene expression both
within hepatocytes and across strains suggests that there may be
differential epigenetic silencing of genes in hepatocytes across
strains. Further, the majority of chromatin states were associated with
variation in expression across strains, suggesting that epigenetic
regulation of gene expression through histone modification may
contribute substantially to variation in gene expression across
genetically distinct individuals. That most states have the same effects
across genes within a cell type and across strains suggests that the
mechanisms that are used to regulate cell type specificity also
contribure to variation in genetically distinct individuals.

\hypertarget{spatial-distribution-of-epigenetic-modifications-around-gene-bodies}{%
\subsection{Spatial distribution of epigenetic modifications around gene
bodies}\label{spatial-distribution-of-epigenetic-modifications-around-gene-bodies}}

In addition to looking for enrichment of chromatin states near annotated
functional elements, we characterized the fine-grained spatial
distribution of each state around gene bodies (Figure XXXA-B). We also
characterized the distribution of CpG sites and their percent
methylation at this gene-level scale (Figure XXXC-D).

The spatial patterns of the individual chromatin states are shown in
(Figure XXXA), and an overlay of all states together emphasizes the
difference in abundance between the most abundant states (states 14, 12,
and 1), and the remaining states, which were relatively rare (Figure
XXXB).

Each chromatin state had a characteristic distribution pattern relative
to gene bodies. For example, state 1, which was characterized by the
absence of all measured histone modifications, was strongly depleted
near the TSS, indicating that this region is commonly subject to histone
modification. However, its abundance increased steadily to a peak at the
TES. In contrast, states 12 and 14 were both concentrated at the TSS.
State 12 was very narrowly concentrated right at the TSS, whereas state
14 was more broadly abundant both upstream and downstream of the TSS.
Both were associated with increased expression in the inbred mice,
suggesting promoter or enhancer functions. The state third state in this
group of high-expressing states, state 13, was depleted nere the TSS,
but enriched within the gene body, suggesting that this state may mark
active intragenic enhancers.

The states with weaker effects on expression, the states with more gray
shades, were also of lower abundance, but still had distinct patterns of
abundance around the gene body suggesting the possibility of distinct
functional roles in the regulation of gene expression.

DNA methylation also showed strong positional effects (Figure XXXC and
D). Across all genes, the TSS had densely packed CpG sites relative to
the gene body (Figure XXXC). As expected, the median CpG site near the
TSS was consistently hypomethylated relative to the median CpG site in
intergenic regions. CpG sites within the gene body were slightly
hypermethylated compared to intergenic CpGs (Figure XXXD).

\hypertarget{spatially-resolved-effects-on-gene-expression}{%
\subsection{Spatially resolved effects on gene
expression}\label{spatially-resolved-effects-on-gene-expression}}

The distinct spatial distributions of the chromatin states and
methylated CpG sites around the gene body raised the question as to
whether the effects of these states on gene expression could also be
spatially resolved. To investigate this possibility we tested the
association between both chromatin state and DNA methylation and gene
expression with spatially resolved models (Methods). We tested the
effect of each chromatin state on expression across genes within
hepatocytes (Figure XXXA) and the effect of each chromatin state on the
variation in gene expression across strains (Figure XXXB).

All chromatin states demonstrated spatially dependent effects on gene
expression within hepatocytes. For about half of the states, the effects
on expression were concentrated at or near the TSS, while in the other
states effects were seen across the whole gene body or predominantly in
the intragenic region. The direction of the effects matched the overall
effects of each state seen previously. Remarkably, the spatial effects
were recapitulated for almost every state when we looked across strains.
That is, variation in chromatin state across strains contributed to
variation in gene expression in the same manner that cell-type
expression was being established. One notable exception was state 9,
whose presence upregulated genes within hepatocytes, but did not
contribute to expression variation across strains.

We also examined the effect of percent DNA methylation across genes
within hepatocytes, and as it contributred to expression variation
across the inbred strains (Figure XXX). As expected, hypomethylation at
the TSS was associated with lower expression in hepatocytes. However,
percent DNA methylation did not contribute at all to expression
variation across strains, implying that although DNA methylation is used
in gene regulation within a cell type, it it not heritable and does not
contribute to variation in gene expression across genetically diverse
individuals.

\hypertarget{imputed-chromatin-state-explained-varation-in-expression-in-diversity-outbred-mice}{%
\subsection{Imputed chromatin state explained varation in expression in
diversity outbred
mice}\label{imputed-chromatin-state-explained-varation-in-expression-in-diversity-outbred-mice}}

Thus far, we have used inbred strains of mice to identify correlations
between local chromatin state and gene expression. However, we cannot
establish causality in this population. For that we need a mapping
population in which we can associate genetic or epigenetic variation at
a single locus with changes in gene expression. A mapping population
will also allow us to establish the extent to which variation in
epigenetic factors contributes to observed expression quantitative trait
loci (eQTL).

To compare the contribution of genetic and epigenetic features to eQTLs
in a gentically diverse population, we imputed chromatin state, DNA
methylation, and SNPs into a population of DO mice described previously
{[}Svenson, Tyler{]} (Methods). Chromatin state is largely determined by
local genotype, especially early in life{[}REF{]}, and can thus be
reliably imputed from local genotype. Further, we have shown here that
local chromatin state correlates with variation in gene expression
across inbred strains. DNA methylation, on the other hand, is known not
to be highly heritable {[}REF{]}, and thus cannot be reliably imputed
from local genotype. We have also shown here that DNA methylation is not
correlated with variation in gene expression across inbred strains. The
imputation of DNA methylation thus serves as an estimate of a lower
bound the ability of a feature imputed from local haplotype to explain
gene expression in a new population.

For each transcript in the DO population, we imputed the local chromatin
state across the gene body based on the gene's local founder haplotype
and the chromatin state at the corresponding position in the inbred
mice. We did the same for DNA methylation and SNPs (Figure XXX).

After imputing each genomic feature into the DO population, we mapped
gene expression to the imputed features and calculated the variance
explained. An example of the data used for the imputation and the
results of the mapping are shown for the gene \textit{Pkd2} in Figure
XXX. Penels B, C, and D show the chromatin state, SNP genotype, and DNA
methylation status at multiple positions along the gene body. The
boundaries of the gene body along with the direction of transcription
are shown by the arrow under panel A. Panel A shows the variance
explained by each genomic feature at each position with a distinct
symbol. The variance explained by the local haplotype is shown as the
dashed line at the top of thie panel. There are two particularly
interesting regions in this gene. One is at the TSS and the immediately
surrounding area, and the other is just downstream of the TSS. Both
regions have high variation in chromatin state that are highly
correlated with gene expression in the DO (blue plus signs in top
panel). Just upstream of the TSS there are SNPs (red x's in the top
panel) that also explain large amounts of variation in expression,
although not as much as chromatin state. A similar pattern is seen at
the internal region just downstream of the TSS. Percent DNA methylation
does not vary across the strains in either of these critical regions,
and does not contribute to variation in expression across genetically
distinct individuals.

The overall distributions of variance explained by each feature across
all transcripts is shown in Figure XXX. These distributions show the
haplotype effect for the marker nearest each transcript compared with
the maximum effect across the gene body for each of the other imputed
features. Overall, local haplotype explained the largest amount of
variance of gene expression in the DO (\(R^2 = 0.17\)). The variance
explained by local chromatin state was very highly correlated with that
of haplotype (Pearson \(r = 0.96\)) and explained almost as much
variance in gene expression in the DO as local haplotype
(\(R^2 = 0.15\)).

The mean variance explained by SNPs was lower (\(R^2 = 0.13\)) than that
explained by haplotype and was not as highly correlated with local
haplotype as chromatin state was (Pearson \(r = 0.93\)). DNA
methylation, the lower bound for variance explained by a feature imputed
from local haplotype, explained the least amount of expression variance
in the DO population (\(R^2 = 0.09\)), and had a much lower correlation
to haplotype than either chromatin state or SNPs (Pearson \(r = 0.74\)).

\hypertarget{discussion}{%
\section{Discussion}\label{discussion}}

In this study we surveyed strain variation in two types of epigenetic
modifications across nine inbred strains of mouse: histone
modifications, and DNA methylation. We identified 14 chromatin states
representing different combinations of four histone modifications and
showed that the presence of these states was associated with variation
in gene expression across the inbred strains. This relationship between
chromatin state and gene expression varied along the gene body and was
concentrated both at the TSS and within the gene body. In addition to
the correlation betweeen chromatin state and gene expression across
inbred strains, we found that chromatin state, imputed from local
founder haplotype, explained a large amount of the variation in gene
expression in an independent outbred mouse population, almost completely
accounting for the variance explained by local eQTLs.

In contrast, although percent DNA methylation was associated with
downregulation of genes within hepatocytes, it was not associated with
gene expression variation across inbred strains or in the outbred
population. Panel D in Figure XXX gives a visual example of why this is
the case. Despite strain variation in both genotype and chromatin state
at the TSS of \textit{Pkd2}, DNA methylation does not vary at all. The
CpG island at the TSS is unmethylated in all strains, and thus all
strains express \textit{Pkd2} in hepatocytes. Rather, it is the
variation in histone modifications that contributes to the variation in
expression levels across the strains.

Similar observations have been made in human studies {[}33931130{]}.
Multiple twin studies have estimated the average heritability of
individual CpG sites to be roughly 0.19 {[}27051996, 24183450,
22532803{]}, with about 10\% of CpG sites having a heritability greater
than 0.5 {[}24183450, 22532803, 24887635{]}. Trimodal CpG sites,
i.e.~those with methylation percent varying among 0, 50, and 100\%, have
been shown in human brain tissue to be more heritable than unimodal, or
bimodal sites (\(h^2 = 0.8 \pm 0.18\)), and roughly half were associated
with local eQTL {[}20485568{]}. Here, we did not see an association
between trimodal CpG sites and gene expression across strains
(Supplemental Figure XXX).

The spatial resolution of chromatin state gives a finer grained picture
of local gene regulation than haplotype without the loss of explanatory
power seen with SNPs. In the example gene \textit{Pkd2}, chromatin state
shows two clear regulatory regions, one surrounding the TSS, and the
other just downstream inside the gene body. These regulatory regions are
not apparent in the SNP patterns or in the patterns or DNA methylation.
The chromatin state map not only helps identify putative regulatory
regions, but may also help annotate the SNPs underyling these regions.
SNPs underlying the two regulatory regions may alter gene expression by
affecting histone modificactions in regulatory regions, whereas the SNPs
that are further downstream of this region may alter gene expression
through other mechanisms, such as through altering transcription itself,
or the processing of the transcript. Thus the intermediate resolution of
the chromatin state between that of SNPs and haplotype provides a highly
informative layer of information between genotype and gene expression.

There were a few individual states that were of particular interest.

more detailed discussion of a few individual states: combine spatial
distribution, spatial effects, literature, and enrichments, we can find
out interesting things about the states and generate hypotheses about
the biology of hepatocytes.

individual states to discuss: states with combo activating and
repressing:

bivalent promoter: typically resolved during development, but may stay
for regulation of some processes? tissue regeneration? abundance is
right at TSS, consistent with promoter annotation. Pretty rare state,
but marks it genes that are enriched for developmental processes, so
probably real. (?) enrichments:ube morphogenesis, tube development,
vasculature development, regulation of locomotion, blood vessel
development, cell adhesion, biological adhesion, regulation of cell
migration, blood vessel morphogenesis, anatomical structure formation,
regulation of cell differentiation

poised enhancer: frequency similar to bivalent promoter. Abundance peaks
upstream of the TSS, so consistent with enhancer annotation.
enrichments: vasculature development, circulatory system development,
cell adhesion, biological adhesion, blood vessel development, regulation
of cell migration, regulation of cell motility, cell-cell adhesion,
protein localization to cell periphery, blood vessel morphogenesis,
locomotion, cell migration, smooth muscle cell differentiation,
regulation of immune system process

both of these states have the same negative regulation of expression at
the TSS in hepatocytes and across inbred strains, but the effect does
not translate to DO mice. What do we make of this?

\textbf{activating states:} active promoter and enhancer have different
distributions, the enhancer is broader around the TSS, the promoter
state is right at the TSS. both have positive effects on transcription
that match their abundance profile. Effects go all the way through to DO
mice

\textbf{active promoter enrichments:} response to cytokine, brush
border, lipid metabolic process, cellular lipid metabolic process, actin
cytoskeleton, cluster of actin-based cell projections, regulation of
intracellular signal transduction, lipid binding, fatty acid metabolic
process, drug transport, brush border membrane, MAPK cascade, positive
regulation of MAPK cascade, monocarboxylic acid metabolic process,
regulation of cell migration, Cortisol synthesis and secretion,
regulation of MAPK cascade, chemical homeostasis

\textbf{active enhancer enrichments:} chromatin binding, tube
morphogenesis, tube development, regulation of establishment of protein
localization, regulation of protein localization, beta-catenin binding,
nucleotide transport, mitochondrial membrane, neuron projection, purine
nucleotide transport, purine ribonucleotide transport, regulation of
cellular component movement, regulation of cell migration, regulation of
anatomical structure morphogenesis

\textbf{intragenic enhancer:} abundance and positive effect peak inside
gene body effect carries all the way through to DO mice enrichments:
cellular macromolecule localization, cellular protein localization,
intracellular transport, establishment of protein localization, protein
transport, RNA processing, cellular response to stress.

\textbf{repressor state:} marks developmental genes

Genes that are marked with these different states are regulated
differently across the inbred strains of mice.

intragenic enhancer that has effect in hepatocytes, but not across
strains: genes in hepatocytes with this state in the gene body are more
highly regulated than genes without this state. The effect is quite
strong in hepatocytes, but does not extend to variation across strains,
even in the inbred mice. This state marks genes that are important in
hepatocytes, but are not differentially regulated across strains.
enrichments: small molecule metabolic process, oxoacid metabolic
process, organic acid metabolic process, carboxylic acid metabolic
process, lipid metabolic process, organic substance catabolic process,
monocarboxylic acid metabolic process

By combining the spatial distribution and effects of the chromatin
states with known information about the effects of histone
modifications, we can

variation in local genotype -\textgreater{} variation in epigenotype
-\textgreater{} variation in expression local eigenetics explains more
variation in expression than individual SNPs - implies that individual
SNPs may not be the best unit of analysis (too polemical?)

for the majority of transcripts we still only explain about 10\% of the
variance with haplotype and local epigenetics other variation explained
by more complex genetic architecture? stochasticity? other genomic
features?

\hypertarget{annotation-of-chromatin-states}{%
\subsection{Annotation of chromatin
states}\label{annotation-of-chromatin-states}}

Make this section into a table

We identified 14 chromatin states corresponding to 14 distinct
combinations of histone modifications (Figure XXXA). To annotate these
states to functional elements, we combined previously known annotations
with functional enrichments and relationship to gene expression. The
characterizations are summarized in Figure XXX. Figure XXXA further
shows the relative abundance of each state in and around the gene body.
This high-resolution image of abundance helped further refine the
annotations of each state. Figure XXXB shows that overall states 1 and 7
were the most abundant states with state 7 being highly enriched at the
TSS, and state 1 being strongly depleted at the TSS, but enriched within
the gene body and in intragenic spaces. We describe the reasoning behind
the annotation of each state below:

\textbf{State 1 - heterochromatin} was characterized by the absence of
all measured marks, enrichment in intergenic regions, and strong
downregulation of gene expression. This state was strongly depleted at
the TSS of expressed genes (Figure XXXA), but the most abundant state in
the gene body and outside the gene body. This state may multiple
different states that could be resolved with the measurement of more
histone modifications. For example, intergenically, state 1 may mark
heterochromatin, which is characterized by H3K9 trimethylation
{[}12867029{]}, which was not measured here. However, state 1 was also
highly abundant in the gene bodies of expressed genes, but was
associated with reduced expression. This could suggest differential
distribution of heterochromatin across strains, or could represent an
additional transcriptionally repressive state.

\textbf{State 2 - repressed chromatin} was characterized by the presence
of H3K27me3, which has been shown previously to correlate with
transcriptional silencing {[}REF{]}. This state was not enriched in any
particular functional element, but was associated with strong
downregulation of transcription.

\textbf{State 3 - poised enhancer} was primarily characterized by the
presence of H3K27me3, a mark associated with polycomb silencing
{[}REF{]}, and H3K4me1 a mark associated with enhancers {[}REF{]}. The
co-occurence of these opposing marks has previously been associated with
a functional element known as a poised enhancer {[}21160473{]}.

This element has been studied mostly in the context of development.
Bivalent promoters are abundant in undifferentiated cells, and are
resolved either to active promoters or silenced promoters as the cells
differentiate into their final state {[}REF{]}. These promoters have
also been shown to be important in the response of cancer cells to
environmental disturbances such as hypoxia {[}REF{]}. The presence of
bivalent promoters in adult mouse hepatocytes is interesting. They may
mark genes poised for expression during liver regeneration, or for
responding to a particular environmental stimulus. There were XXX genes
that were marked with this bivalent promoter state at the TSS across all
strains. This group of genes was enriched for developmental processes as
well as alcohol metabolism (Fig? Table?).

\textbf{State 4 - intragenic enhancer} was characterized by the presence
of H3K4me1, which is known to mark cell type-specific enhancers, both
active and poised {[}REF{]}. The presence of H3K4me1 alone, in the
absence of H3K27ac, as it occurs in state 4, has been shown to mark
inactive, or poised enhancers {[}21106759{]}. The addition of H3K27ac
can then activate the enhancer to increase transcription. When present
within the gene body, this state acts as an intragenic enhancer, which
acts as an alternative promoter, and can be transcribed bidirectionally
to produce short RNAs known as eRNA {[}20393465{]}. This state was
modestly enriched in known enhancers and was associated with slightly
increased gene expression. The presence of H3K4me1 in the absence of
H3K4me3 has been shown to mark intragenic enhancers and to be associated
with increased transcription, as these regions can be transcribed
independently of the full gene {[}Kowalczyk et al.~2012{]}. We annotated
this state as a weak enhancer.

\textbf{State 5 - active enhancer} was characterized by the co-occurence
of H3K4me1, which marks cell type-specific enhancers, and H3K27ac, which
specifically marks active enhancers {[}21106759, 21160473{]}. This state
was strongly enriched in known enhancers, and its presence had a strong
postive effect on transcription. We thus annotated this state as a
strong enhancer.

state 0100 (just H3K4me3 absence of others) - interesting abundance
pattern. Just up and downstream of the TSS, and just downstream of the
TES. Associated with slight downregulation of expression. Could be
antisense transcription {[}22768981{]}? This fits the negative
association, but the position of the negative association is just
downstream of the TSS. can't really find any other sources on this. The
probably with most studies is that they are looking only at a single
mark, and not noting the absence of the other three marks.

state 0110 -

\hypertarget{acknowledgements}{%
\section{Acknowledgements}\label{acknowledgements}}

This work was funded by XXX.

\hypertarget{data-and-software-availability}{%
\section{Data and Software
Availability}\label{data-and-software-availability}}

All data used in this study and the code used to analyze it are avalable
as part of a reproducible workflow located at\ldots{} (Figshare?,
Synapse?).

\hypertarget{supplemental-figure-legends}{%
\section{Supplemental Figure
Legends}\label{supplemental-figure-legends}}

\begin{figure}[ht]
\centering
\caption{
}
\label{fig:trait_cor}
\end{figure}

\hypertarget{supplemental-table-descriptions}{%
\section{Supplemental Table
Descriptions}\label{supplemental-table-descriptions}}

\begin{figure}[ht]
\centering
\caption{Correlations between traits and the first PC of the kinship matrix.
}
\label{table:trait_cor}
\end{figure}

\hypertarget{references}{%
\section*{References}\label{references}}
\addcontentsline{toc}{section}{References}

\nolinenumbers


\end{document}
